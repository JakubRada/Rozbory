\documentclass[11pt]{article}
\usepackage[a4paper, total={18cm, 27cm}]{geometry}
\usepackage[czech]{babel}
\usepackage[utf8]{inputenc}

\begin{document}
    \begin{center}
        \underline{\textbf{\Huge Karel Hynek Mácha - Máj}}
    \end{center}
    \section*{Úryvek}
    Kde za jezerem hora horu

    v západní stíhá kraje,

tam – zdá se mu – si v temném boru

posledně dnes co dítko hraje.

Od svého otce v svět vyhnán,

v loupežnickém tam roste sboru.

Později vůdcem spolku zván,

dovede činy neslýchané,

všude jest jméno jeho znané,

každémuť: „Strašný lesů pán!“

Až poslez láska k růži svadlé

nejvejš roznítí pomstu jeho,

a poznav svůdce dívky padlé

zavraždí otce neznaného.

Protož jest u vězení dán;

a kolem má být odpraven

již zítra strašný lesů pán,

jak první z hor vyvstane den.

\bigskip

Teď na kamenný složen stůl

hlavu o ruce opírá,

polou sedě a kleče půl

v hloub myšlenek se zabírá;

po měsíce tváři jak mračna jdou,

zahalil vězeň v ně duši svou,

myšlenka myšlenkou umírá.

\bigskip

„Sok – otec můj! Vrah – jeho syn,

on svůdce dívky mojí! –

Neznámý mně. – Strašný můj čin

pronesl pomstu dvojí.

Proč rukou jeho vyvržen

stal jsem se hrůzou lesů?

Čí vinu příští pomstí den?

Čí vinou kletbu nesu?

Ne vinou svou! – V života sen

byl jsem já snad jen vyváben,

bych ztrestal jeho vinu?

A jestliže jsem vůli svou

nejednal tak, proč smrtí zlou

časně i věčně hynu? –

Časně i věčně? – věčně – čas –“

Hrůzou umírá vězně hlas

obražený od temných stěn;

hluboké noci němý stín

daleké kobky zajme klín,

a paměť vězně nový sen.
    \section*{Analýza uměleckého textu}
    \begin{itemize}
        \item\textbf{zasazení výňatku do kontextu díla}
        \begin{itemize}
            \item 1. polovina díla, 2. zpěv, před tím se Jarmila dozvěděla, že Vilém bude popraven, potom je první mezizpěv
        \end{itemize}
        \item\textbf{téma a motiv}
        \begin{itemize}
            \item\textbf{téma: }loučení se se zemí, smysl života, láska, obžaloba společnosti (rozvrácená rodina, bezcitné mezilidské vztahy), osudovost
            \item\textbf{motivy: }poprava, pomsta, stěžování si na osud, země jako matka, příroda, kontrast krásné přírody a kruté společnosti
        \end{itemize}
        \item\textbf{časoprostor}
        \begin{itemize}
            \item\textbf{čas: }blíže neurčený, na jaře - rozkvět přírody; noc a ráno před popravou Viléma, pak po 7 letech
            \item\textbf{prostor: }příroda pod Bezdězem - příroda mu pomáhá, je jí ho líto - kontrast osudu, jaro
        \end{itemize}
        \item\textbf{kompoziční výstavba}
        \begin{itemize}
            \item rozděleno do 4 zpěvů a 2 mezizpěvů (rej duchů, očekávají mrtvého; truchlení loupežníků nad jejich pánem) (1-2-a-3-b-4)
            \item předzpěv - oslava Čechů (jsou národ dobrý), aby se trochu zavděčil vlastencům, protože Máj není podle nich dostatečně vlastenecký
            \item hodně opakujících se motivů - příroda, smrt
            \item gradace k vrcholu díla - 3. zpěv
        \end{itemize}
        \item\textbf{literární žánr a druh}
        \begin{itemize}
            \item\textbf{druh: }lyricko-epický
            \item\textbf{žánr: }poema
        \end{itemize}
        \item\textbf{vypravěč / lyrický subjekt}
        \begin{itemize}
            \item lyrickým subjektem je Hynek - ztotožňuje se s Vilémem, cítí nezastavitelnost času, přichází na místo dění po 7 letech
            \item er-forma, poslední zpěv ich
        \end{itemize}
        \item\textbf{postava}
        \begin{itemize}
            \item Vilém - fiktivní, přirozená, hlavní - rekapituluje svůj život, nelituje vraždy otce, který svedl Jarmilu, jeho milou, loupežník (strašný lesa pán), přemýšlivý, romantický hrdina, vnitřně rozervaný
        \end{itemize}
        \item\textbf{typy promluv}
        \begin{itemize}
            \item přímá i nepřímá řeč - uvozovky
        \end{itemize}
        \item\textbf{veršová výstavba}
        \begin{itemize}
            \item vázaný verš - jamb (poprvé v české poezii)
            \item střídá typy (střídavý, obkročmý, \dots)
        \end{itemize}
        \item\textbf{jazykové prostředky a jejich funkce ve výňatku}
        \begin{itemize}
            \item stará spisovná čeština
            \begin{itemize}
                \item knižní - každémuť, jest
                \item přechodníky - poznav
                \item využívá dvojhlásky - ou, au - zní to lépe
            \end{itemize}
        \end{itemize}
        \item\textbf{tropy a figury a jejich funkce ve výňatku}
        \begin{itemize}
            \item inverze
            \item anafora - čí, čí
            \item personifikace - den vyvstane
        \end{itemize}
    \end{itemize}
    \section*{Literárně historický kontext a autor}
    Karel Hynek Mácha byl jediný český typický romantik.
    Vystudoval práva, hodně cestoval, hlavně příroda a zříceniny - inspirace do děl.
    Typičtí romantičtí hrdinové.
    Máj je jeho vrcholné dílo.
    Psal prózu - Noc na Bezdězu, Márinka, Cikáni, Křivoklát.
    Byl to vrchol národního obrození - byl kritizován, že Máj není dostatečně vlastenecký a že plýtvá svým talentem + říkal, že po smrti nic není.
    Tvořil v 1. polovině 19. století, jediný představitel romantismu u nás.
    Další autoři u nás: Erben, Borovský, Němcová; ve světě: Byron, Schiller, Puškin.
    Dobově dílo nepochopeno, kritizováno. Navazují na něj májovci, dnes považováno za velmi důležité.
\end{document}