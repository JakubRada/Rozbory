\documentclass[11pt]{article}
\usepackage[a4paper, total={18cm, 27cm}]{geometry}
\usepackage[czech]{babel}
\usepackage[utf8]{inputenc}

\begin{document}
    \begin{center}
        \underline{\textbf{\Huge }}
    \end{center}
    \section*{Úryvek}
    \section*{Analýza uměleckého textu}
    \begin{itemize}
        \item\textbf{zasazení výňatku do kontextu díla}
        \begin{itemize}
            \item
        \end{itemize}
        \item\textbf{téma a motiv}
        \begin{itemize}
            \item
        \end{itemize}
        \item\textbf{časoprostor}
        \begin{itemize}
            \item
        \end{itemize}
        \item\textbf{kompoziční výstavba}
        \begin{itemize}
            \item
        \end{itemize}
        \item\textbf{literární žánr a druh}
        \begin{itemize}
            \item
        \end{itemize}
        \item\textbf{vypravěč / lyrický subjekt}
        \begin{itemize}
            \item
        \end{itemize}
        \item\textbf{postava}
        \begin{itemize}
            \item
        \end{itemize}
        \item\textbf{vyprávěcí způsoby}
        \begin{itemize}
            \item
        \end{itemize}
        \item\textbf{typy promluv}
        \begin{itemize}
            \item
        \end{itemize}
        \item\textbf{veršová výstavba}
        \begin{itemize}
            \item
        \end{itemize}
        \item\textbf{jazykové prostředky a jejich funkce ve výňatku}
        \begin{itemize}
            \item
        \end{itemize}
        \item\textbf{tropy a figury a jejich funkce ve výňatku}
        \begin{itemize}
            \item
        \end{itemize}
    \end{itemize}
    \section*{Literárně historický kontext a autor}
\end{document}