\documentclass[11pt]{article}
\usepackage[a4paper, total={18cm, 27cm}]{geometry}
\usepackage[czech]{babel}
\usepackage[utf8]{inputenc}

\begin{document}
    \begin{center}
        \underline{\textbf{\Huge Ernest Hemingway - Stařec a moře}}
    \end{center}
    \section*{Úryvek}
    "Pohnul jsem s ní," řekl si stařec nahlas. "Přece jsem s ní pohnul!"
Pocítil opět nával mdloby, ale visel na obrovské rybě, co měl síly.
Pohnul jsem s ní, opakoval si v duchu. Možná, že teď ji převrátím. Táhněte, ruce! Držte, nohy!Hlavo, vydrž! Udělej mi to kvůli! Nikdy jsi mě nezklamala. Tentokrát ji převrátím na bok!
Když však sebral všechny síly, opřel se do toho o hodně dřív, než ryba proplula kolem člunu, a táhl, div mu svaly nepraskly, ryba se dala o něco strhnout, ale pak vyrovnala a plula pryč.
"Rybo," oslovil ji stařec. "Rybo, vždyť stejně umřeš. Musíš přitom zabít i mne?"
Takhle nic nedokážu, napadlo ho. Měl v ústech příliš vyschlo, aby promluvil nahlas, ale pro vodu teď nemohl sáhnout. Musím ji dostat po bok loďky tentokráte, umiňoval si v duchu. Moc už těch obrátek nevydržím. Ale vydržíš! okřikl sám sebe. Vydržíš to do nekonečna.
    \section*{Analýza uměleckého textu}
    \begin{itemize}
        \item\textbf{zasazení výňatku do kontextu díla}
        \begin{itemize}
            \item 2.polovina knihy i lovu, už je unavený po dlouhém držení obrovské ryby na provaze, těsně před tím, než ji konečně zabije harpunou
        \end{itemize}
        \item\textbf{téma a motiv}
        \begin{itemize}
            \item\textbf{téma: }síla lidské vůle a odhodlání (i přes obrovské vyčerpání a stáří se Santiago nevzdává a bojuje s rybou do posledních sil); nekonečný boj člověka s přírodou - příroda je vždy silnější a nelze ji ovládnout, ale člověk je na ní závislý; smysl života (smyslem není zisk, ale snaha dosáhnout nějakého cíle)
            \begin{itemize}
                \item „Člověka je možno zničit, ale ne porazit.“
            \end{itemize}
            \item\textbf{motivy: }lov (Hemingwayova láska k lovu), kontrast - stáří a únava $x$ mládí a síla, příroda
        \end{itemize}
        \item\textbf{časoprostor}
        \begin{itemize}
            \item vesnice na Kubě nedaleko Havany ve 40. letech 20. století - reálný svět na konkrétním místě
            \begin{itemize}
                \item napsal to v 50. letech po několika letech strávených na Kubě, kde se věnoval i rybařině - to prostředí mu bylo blízké, i trochu zkušenost
                \item tři dny rybaření na moři - je sám a kolem něj je široko daleko jen voda - člověk je bezmocný oproti přírodě
            \end{itemize}
        \end{itemize}
        \item\textbf{kompoziční výstavba}
        \begin{itemize}
            \item není členěno na kapitoly, je to jeden souvislý text, pouze odstavce
            \item chronologické vyprávění, jedna nepřerušovaná příběhová linie - sledujeme pouze dění kolem Santiaga
        \end{itemize}
        \item\textbf{literární žánr a druh}
        \begin{itemize}
            \item\textbf{druh: }epika (s lyrickými prvky), próza
            \item\textbf{žánr: }novela, lyrická novela
        \end{itemize}
        \item\textbf{vypravěč}
        \begin{itemize}
            \item nadosobní vševědoucí vypravěč, nedává vlastní názor na děj a jednání postav
            \item er-forma
        \end{itemize}
        \item\textbf{postava}
        \begin{itemize}
            \item rybář Santiago - hlavní, fiktivní, přirozená postava, spíše statická - starý rybář, který už má 84 dní smůlu a nechytil žádnou větší rybu, kvůli tomu ho musel opustit chlapec, který byl jeho přítelem a kterého naučil rybařit; 85. den se mu podaří chytit chytit obrovskou rybu a 2 dny a 2 noci s ní svádí nerovný souboj, velmi vytrvalý se silnou vůlí - i přes obrovské vyčerpání nakonec rybu zabije, smůla ho bohužel neopouští a celou rybu mu sežerou žraloci a vrací se do vesnice pouze s její kostrou; ani to ho ale nezlomí a plánuje další lov s chlapcem
                \begin{itemize}
                    \item Hemingway tvrdí, že nemá předlohu, ale někteří v něm vidí jeho přítele rybáře z Kuby Gregorio Fuentes
                \end{itemize}
            \item chlapec Manoli - vedlejší, fiktivní, přirozená, statická postava - mladý, má rád Santiaga, naučil ho rybařit, povídají si spolu o baseballu - symbolizuje mezigenerační souznění
        \end{itemize}
        \item\textbf{vyprávěcí způsoby}
        \begin{itemize}
            \item er-forma
        \end{itemize}
        \item\textbf{typy promluv}
        \begin{itemize}
            \item přímá (někdy nevlastní - \textit{"Možná, že teď ji převrátím. Táhněte, ruce! Držte, nohy!Hlavo, vydrž! Udělej mi to kvůli! Nikdy jsi mě nezklamala. Tentokrát ji převrátím na bok!"}) i nepřímá řeč - časté vnitřní monology a promluvy k sobě samému
        \end{itemize}
        \item\textbf{jazykové prostředky a jejich funkce ve výňatku}
        \begin{itemize}
            \item spisovný jazyk
            \item ponechány některé španělské výrazy - lepší navození prostředí
            \item krátké strohé věty, krátká slova
            \item metoda ledovce - psaný text je pouze malá část (1/8) významu celého díla, snaží se co nejméně slovy vystihnout co nejvíce významu
        \end{itemize}
        \item\textbf{tropy a figury a jejich funkce ve výňatku}
        \begin{itemize}
            \item jazyk je strohý, ne moc tropů ani figur
            \item křesťanské symboly - stigmata na rukách od provazu, provaz jako kříž (nejsou v ukázce)
            \item metafora - visel na obrovské rybě
            \item hyperbola - táhl div mu svaly nepraskly
        \end{itemize}
    \end{itemize}
    \section*{Literárně historický kontext a autor}
    Narodil se v Americe, ale velkou část strávil v Evropě - Francie, Španělsko, Itálie - působil jako řidič sanitky Červeného kříže za WWI.
    Pak působil jak novinář a dokumentarista např. ve španělské občanské válce, z čehož plyne inspirace pro některá jeho díla (Komu zvoní hrana, Sbohem armádo).
    Hodně času strávil na Kubě - inspirace pro Stařec a moře.
    Ke konci života trpěl depresemi a nakonec spáchal sebevraždu.
    Hemingwayovský hrdina - hrdina je vržen do nějaké krizové situace s vědomím blízké smrti, mužný, toužící po pravdě, snažící se žít důstojně ve zlém světě. (Robert Jordan, Komu zvoní hrana)
    Stařec a moře je jeho poslední významné dílo, obdržel za něj Nobelovu cenu.
    Psal hlavně mezi válkami a pak po WWII. Jeho nejdůležitější díla jsou Stařec a moře, Sbohem armádo, Fiesta a Komu zvoní hrana.
    Patří ke Ztracené generaci a sám tento termín používal. Jedná se o americké autory, kteří žili v Paříži ve 20. letech a které poznamenala WWI.
    Mimo něj ke Ztraceno generaci patřil např. Francis Scott Fitzgerald a Eliot.
    Dílo bylo dobře přijato, díla byla několikrát zfilmována.
\end{document}