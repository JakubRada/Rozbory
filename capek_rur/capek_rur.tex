\documentclass[11pt]{article}
\usepackage[a4paper, total={18cm, 27cm}]{geometry}
\usepackage[czech]{babel}
\usepackage[utf8]{inputenc}

\begin{document}
    \begin{center}
        \underline{\textbf{\Huge Karel Čapek - R.U.R.}}
    \end{center}
    \section*{Úryvek}
    DR. GALL (vejde): Dobré jitro, paní Dominová. Co máte pěkného? \\ HELENA: Tady Radia, doktore. \\ DR.  GALL:  Aha,  náš  chlapík  Radius.  Tak  co,  Radie,  děláme pokroky? \\ HELENA: Ráno měl záchvat. Rozbíjel sochy. \\ DR. GALL: Kupodivu, on také? \\ HELENA: Jděte, Radie! \\ DR.  GALL:  Počkat!  (Otočí  Radia  k  oknu,  zakrývá  a  odkrývá  mu dlaní  oči,  pozoruje  reflexy  zorniček.)  Koukejme.  Prosím  jehlu. Nebo špendlík. HELENA (podává mu jehlici): Nač to? \\ DR.  GALL:  Jen  tak.  (Bodne  Radia  do  ruky,  jež  prudce  ucukne.) Pomalu, hochu. Můžete jít. RADIUS: Děláte zbytečné věci. (Odejde.) \\ HELENA: Co jste s ním dělal? \\ DR.  GALL  (usedne):  Hm,  nic.  Zorničky  reagují,  zvýšená  citlivost  a tak dále. – Oho! tohle nebyla křeč Robotů! \\ HELENA: Co to bylo? \\ DR. GALL: Čert ví. Vzdor, zuřivost nebo vzpoura, já nevím co. \\ HELENA: Doktore, má Radius duši? \\ DR. GALL: Nevím. Má něco ošklivého. \\ HELENA: Kdybyste věděl, jak nás nenávidí! Oh Galle, jsou všichni vaši Roboti takoví? Všichni, které jste... začal dělat... jinak? \\ DR. GALL: Inu, jsou jaksi vznětlivější – Co chcete? Jsou podobnější lidem než Roboti Rossumovi. 
\\ HELENA: Je snad i ta... nenávist podobnější lidem? \\ DR. GALL (krčí rameny): I ta je pokrok. \\ HELENA: Kam se poděl ten váš nejlepší – jak se jmenoval?\\ DR. GALL: Robot Damon? Toho prodali do Havru. \\ HELENA: A vaše Robotka Helena? \\ DR. GALL: Váš miláček? Ta mně zůstala. Je rozkošná a hloupá jako jaro. Jednoduše není k ničemu. \\ HELENA: Vždyť je tak krásná! \\ DR. GALL: Což vy víte, jak je krásná? Z rukou božích nevyšlo dílo dokonalejší,  než  je  ona!  Chtěl  jsem,  aby  byla  podobna  vám  – Bože, jaký nezdar! \\ HELENA: Proč nezdar? \\ DR.  GALL:  Protože  není  k  ničemu.  Chodí  jako  ve  snu, rozviklaná, neživá – Bože můj, jak může být krásná, když nemiluje? Dívám se na ni a hrozím se, jako bych mrzáka stvořil. Ach Heleno, Robotko Heleno,  nikdy  tedy  tvé  tělo  neoživne,  nebudeš  milenkou, nebudeš  matkou;  tyhle  dokonalé  ruce  si  nebudou  hrát  se zrozeňátkem, neuvidíš svou krásu v kráse svého dítěte –HELENA (zakrývá si tvář): Oh mlčte! \\ DR.  GALL:  A  někdy  si  myslím:  Kdybys  procitla,  Heleno,  jen  na okamžik, ach jak bys vykřikla hrůzou! Snad bys zabila mne, který jsem tě stvořil; snad bys vrhla slabou rukou kámen do těch strojů tady, které rodí Roboty a zabíjejí ženství, nešťastná Heleno! \\ HELENA: Nešťastná Heleno! \\ DR. GALL: Co chcete? Není k ničemu. (Pauza.) \\ HELENA: Doktore – \\ DR. GALL: Ano. \\ HELENA: Proč se přestaly rodit děti? \\ DR. GALL: – – Nevíme, paní Heleno. \\ HELENA: Povězte mi to! 
 \\ DR. GALL: Protože se dělají Roboti. Protože je nadbytek pracovních sil. Protože člověk je vlastně přežitek. Vždyť to už je, jako by se – – eh! \\ HELENA: Řekněte to.  GALL: – jako by se příroda výrobou Robotů urazila. \\ HELENA: Galle, co se stane s lidmi?  GALL: Nic. Proti přírodě se nedá nic dělat. \\ HELENA: Proč Domin neomezí –  GALL: Odpusťte, Domin má své ideje. Lidem, kteří mají ideje, by se neměl dávat vliv na věci tohoto světa. \\ HELENA: A žádá někdo, aby se... vůbec přestalo vyrábět? \\ DR. GALL: Bůh uchovej! Ten by si dal! \\ HELENA: Proč? \\ DR. GALL: Protože by ho lidstvo ukamenovalo. Víte, je to přece jen pohodlnější, nechat za sebe pracovat Roboty. HELENA  (vstane):  A  řekněte,  kdyby  někdo  rázem  zastavil  výrobu Robotů – \\ DR. GALL (vstane): Hm, to by byla pro lidi strašná rána. \\ HELENA: Proč rána? \\ DR. GALL: Protože by se musili vrátit tam, kde bývali. Ledaže by – \\ HELENA: Řekněte. \\ DR. GALL: – ledaže by bylo už na návrat pozdě. 
    \section*{Analýza uměleckého textu}
    \begin{itemize}
        \item\textbf{zasazení výňatku do kontextu díla}
        \begin{itemize}
            \item ukáza z prvního dějství - přibližně polovina knihy, poté Helena přemlouvá Dr. Galla, aby exprimentoval s robotí duší - skončí to vyhubením lidstva kromě Alquista
        \end{itemize}
        \item\textbf{téma a motiv}
        \begin{itemize}
            \item\textbf{téma: }nebezpečí technického rozvoje pro lidstvo, pokrok lidstvo ovládá, jediná záchrana je láska, kritizuje lidskou touhu příliš zlenivět
            \item\textbf{motivy: }láska, na motivy biblického příběhu o Adamovi a Evě, válka, zlenivění lidstva kvůli strojům
        \end{itemize}
        \item\textbf{časoprostor}
        \begin{itemize}
            \item\textbf{čas: }blíže neurčená budoucnost, přesný čas nehraje roli
            \item\textbf{prostor: }reálný svět na vymyšleném místě, ostrov společnosti Rossumovi univerzální roboti - zaměstnanci továrny zůstanou z lidstva jako poslední - jistá symbolika (začalo to u nich, tak to tam i skončí)
        \end{itemize}
        \item\textbf{kompoziční výstavba}
        \begin{itemize}
            \item dílo rozděleno do předehry a tří dějství, na začátku výčet a krátká charakteristika postav
            \item dějství už nejsou děleny na scény
            \item chronologické
        \end{itemize}
        \item\textbf{literární žánr a druh}
        \begin{itemize}
            \item\textbf{druh: }drama
            \item\textbf{žánr: }vědeckofantastické drama, utopistické
        \end{itemize}
        \item\textbf{vypravěč / lyrický subjekt}
        \begin{itemize}
            \item vypravěč není, je to drama, pozůstatek vypravěče ve scénických poznámkách
        \end{itemize}
        \item\textbf{postava}
        \begin{itemize}
            \item Harry Domin - generální ředitel, chce povznést člověka nad dřinu, proto vyrábí roboty podle tajemství starého Rossuma
            \item Helena - manželka Domina, ztělesnění lidského, laskavého přístupu k robotům a lidem
            \item Alquist - vrchní stavitel, rozumový přístup, za nejvyšší hodnotu považuje lidskou práci, roboti po něm chtějí recept na výrobu dalších robotů, při pitvě jednoho z nich objeví, že se nechovají logicky - mají emoce, vidí jak se do sebe dva roboti zamilují - Adam a Eva
            \item Dr Gall - velitel fyziologického a vědeckého výzkumu, na Heleninu žádost zkoumá duši robotů
            \item všechny fiktivní, přirozené
        \end{itemize}
        \item\textbf{vyprávěcí způsoby}
        \begin{itemize}
            \item du-forma, repliky, dialogy, monology (hlavně Alquist, když už zbývá jako jediný člověk)
        \end{itemize}
        \item\textbf{typy promluv}
        \begin{itemize}
            \item je to drama - repliky skládané do monologů nebo dialogů
        \end{itemize}
        \item\textbf{jazykové prostředky a jejich funkce ve výňatku}
        \begin{itemize}
            \item skoro jenom spisovný jazyk (Helenina komorná Nana - obecná čeština)
            \item krátké věty - lépe simulují skutečný dialog
        \end{itemize}
        \item\textbf{tropy a figury a jejich funkce ve výňatku}
        \begin{itemize}
            \item přirovnání - hloupá jako jaro
        \end{itemize}
    \end{itemize}
    \section*{Literárně historický kontext a autor}
    Karel Čapek je jeden z nejznámějších a nejvýznamnějších českých spisovatelů (významný fejetonista), působil jako redaktor Národních listů a Lidových novin.
    Vytvořil novinový sloupek. Je představitelem demokratického proudu po WWI.
    Spoluzaložil spolek pátečníci - on, bratr, Peroutka, TGM, Beneš.
    RUR je z jeho druhého, utopického období - upozorňuje na nebezpečí pokroku a vědy, ale i z fašismu. Považován za předchůdce sci-fi literatury.
    Mezi další významná díla patří např. Krakatit, Bílá nemoc, Válka s mloky, Hovory s TGM.
    Tvořil převážně v meziválečném období v 1. polovině 19. století.
    Spolutvořil i se svým bratrem Josefem - Pejsek a kočička, Dášeňka, neboli život štěněte, Ze života hmyzu.
    Obrovský úspěch jeho děl, sedmkrát nominován na Nobelovu cenu.
    Byla krátce zfilmována BBC, vytvořena byla i opera. Přeložena do mnoha jazyků.
\end{document}