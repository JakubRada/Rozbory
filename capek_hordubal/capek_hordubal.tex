\documentclass[11pt]{article}
\usepackage[a4paper, total={18cm, 27cm}]{geometry}
\usepackage[czech]{babel}
\usepackage[utf8]{inputenc}

\begin{document}
    \begin{center}
        \underline{\textbf{\Huge Karel Čapek - Hordubal}}
    \end{center}
    \section*{Úryvek}
    A teď neví Juraj Hordubal, co by řekl: tolik začátků si myslil, co to, že se sem žádný nehodí? Nezakryje dlaněmi Polaně, nezaťuká v noci na okno, nevrátí se se zvoněním stád a se slovy požehnání; ale ježatý sem vráží a neumytý, nu, jaký div, že se ženská poleká? I hlas by ze mne vyšel divný a seškrcený –poraď, Hospodine, co se dá říci takovým světu nepodobným hlasem.Polana ustupuje ze zápraží, až příliš daleko ustupuje, ach, Polano, já bych byl proklouzl i tak; a říká hlasem, který skoro není hlas a skoro není její: „Pojď dál, já –zavolám Hafii.” Ano, Hafii, ale dřív bych chtěl tobě položit ruce na ramena a říci, nu, Polano, nerad jsem tě polekal; sláva Bohu, že už jsem doma. A vida, jak jsi se zařídila: nová je postel a vysoko nastlaná, stůl nový a těžký, na stěně svaté obrazy, inu, brachu, to ani v Americe nemají lepší; podlaha z prken a v oknech muškát, dobře jsi, Polano, gazdovala! Juraj Hordubal potichoučku usedá na svůj kufřík. Moudrá je Polana a ví si rady; jak to tu vyzírá, hádal bys, že má dvanáct krav, dvanáct i víc –Chvála Bohu, ne nadarmo jsem robotil; ale to horko v shaftu, dušinko, kdybys věděla, jaké peklo!Polana nejde; Juraji Hordubalovi je nějak tísnivě jako tomu, kdo sám a sám je v cizí jizbě. Počkám na dvorku, říká si, snad abych se zatím umyl. Joj, svléknout košili  a  napumpovat  si studené vody na plece, na hlavu, do vlasů, přehojně stříkat vodou kolem a zařičet libostí, hej! ale to by se jaksi nehodilo, ještě ne, ještě ne; jen trochou vody z železné pumpy (bývalo tu dřevěné roubení a okov na vahadle, a ta hlubokátma dole, a jak to vlhce a chladně dechlo, když se člověk naklonil nad roubení) (toto je jako v Americe, tam mívají farmáci takové pumpy) (s plným okovem do chléva a napájet krávy, až se jim nozdry lesknou vlhkostí a 
    hlasitě  frkají),  jen  trochou  vody  smáčí  umolousaný  kapesník  a  utírá  si  čelo,  ruce,  týl, achachach, to chladí, ždímá kapesník a hledá, kam jej pověsit, ještě ne, ještě tu nejsme doma, a mokrý jej cpe do kapsy.„Tady máš tatínka, Hafie,” slyší Hordubal, a Polana k němu strká jedenáctileté děvčátko poplašených, bledě modrých očí. „Tak ty jsi Hafie,” bručí Hordubal rozpačitě (jej Bohu, pro tak velké dítě teddy-bear!) a chce jí sáhnout po vlasech, jen tak prstem, Hafie; ale děvčátko uhýbá, tiskne se k matce a nespouští očí z toho cizího mužského. „Tak pozdrav přece, Hafie,” praví Polana tvrdě a žďuchá holčičku do zad. Ach, Polano, nech ji –co na tom, že se dítě poleká! „Dobrý den,” šeptá Hafie a odvrací se. Jurajovi je náhle nějak divně, oči se mu zalily, tvář dítěte se před ním třese a rozplývá; ale, ale, copak to –e nic, to nic není, jen že jsem už tolik let neslyšel ,dobrý den‘. „Pojď se podívat, Hafie,” praví honem, „co jsem ti přivezl.”„Jdi, hloupá,” strká ji Polana.Hordubal  klečí  u  kufru,  matičko  boží,  to  se  to  všechno  cestou  zmuchlalo,  a  hledá elektrickou baterku, to se bude Hafie divit! „Tak vidíš, Hafie, tady stiskneš ten gombík, a svítí to.” Nu copak, copak, ono to nechce svítit; Hordubal mačká na knoflík, obrací baterku ze všech stran a zesmutní. „Copak se s tím stalo? Aha, to asi vyschlo tam uvnitř, co je ta elektrika, –víš, bylo tak horko na lowerdecku.Inu, jasně to svítilo, Hafie, jako sluníčko. Ale počkej, já jsem ti přinesl obrázky, to něco uvidíš!” Hordubal loví z kufříku listy z magazínů a novin,  kterými  proložil  těch  pár  kousků šatstva. „Pojď  sem,  Hafie,  ať  vidíš,  jak  vypadá Amerika.”
    \section*{Analýza uměleckého textu}
    \begin{itemize}
        \item\textbf{zasazení výňatku do kontextu díla}
        \begin{itemize}
            \item úplný začátek knihy, kdy se Hordubal vrátil z Ameriky ke své ženě a dceři, která ho nezná
            \item postupně mu začne být podezřelý čeledín Manya, protože ho s ním Polana podváděla
        \end{itemize}
        \item\textbf{téma a motiv}
        \begin{itemize}
            \item\textbf{téma: }způsob a rozsah lidského poznání pravdy, poznání pravdy záleží na člověku, může to být velmi složité, relativnost pravdy (vidíme pravdu ze 3 pohledů - Hordubal, četník, svědkové)
            \item\textbf{motivy: }nevěra, vražda, láska, podlost, chtivost, podle skutečného příběhu
        \end{itemize}
        \item\textbf{časoprostor}
        \begin{itemize}
            \item\textbf{čas: }blíže neurčený, po několika letech v zahraničí se vrací domů, období po WWI (Československo získalo Podkparpatskou Rus)
            \item\textbf{prostor: }vesnice Krivá v Podkparpatské Rusi, reálný, konkrétní
        \end{itemize}
        \item\textbf{kompoziční výstavba}
        \begin{itemize}
            \item dělí se na tři knihy (v první se odehrává hlavní děj, v druhé je vyšetřování Hordubalovy vraždy, ve třetí je pak soud s manželkou a čeledínem)
            \item první 2 knihy se dělí na kapitoly (24, 6), třetí se už nedělí
            \item první je chronologická celá, druhá a třetí mají retrospektivní pasáže, kdy se spekuluje, jak proběhla vražda
        \end{itemize}
        \item\textbf{literární žánr a druh}
        \begin{itemize}
            \item\textbf{druh: }epika, próza
            \item\textbf{žánr: }novela
            \item někdy označováno za baladu v próze
        \end{itemize}
        \item\textbf{vypravěč / lyrický subjekt}
        \begin{itemize}
            \item vypravěč reflektor - vidíme svět skrz vnímání postavy a jeho vnímání, ale přes prostředníka
            \item subjektivizovaná er-forma
        \end{itemize}
        \item\textbf{postava}
        \begin{itemize}
            \item Juraj Hordubal - fiktivní postava, přirozená, neutrální (je obětí, ale nepředstavuje úplně dobro) - manžel Polany, otec Hafie, kterou nepoznal, protože odjel pracovat na 8 let do Ameriky do dolů, vrací se, chladné přijetí od své ženy - naivně si to idealizoval, i když slyší od několika lidí zvěsti o Polanině nevěře s Manyou, pořád jí věří, že byla věrná, i když je vidí jak se objímají mimo vesnici, pořád se nechá přesvědčit; ikdyž Manyu několikrát vyhodí ze statku, vždycky ho pak vezme zpátky; je zabit Manyou
            \item Polana - fiktivní postava, přirozená - nevěrná manželka Hordubala, chladná, nevíme jestli se líbila jenom Hordubalovi, nebo jestli je opravdu hezká, je vždy rozzlobená, když Hordubal vyhodí Manyu ze statku, pomůže mu zabít Hordubala
            \item Štěpán Manya - fiktivní, přirozená postava - nově najatý čeledín, mění zaběhlé pořádky na statku - kupuje koně místo pěstování na místních polích, zabije Hordubala - nevíme jestli kvůli Polaně, nebo kvůli penězům
        \end{itemize}
        \item\textbf{vyprávěcí způsoby}
        \begin{itemize}
            \item subjektivizovaná er-forma
        \end{itemize}
        \item\textbf{typy promluv}
        \begin{itemize}
            \item přímá řeč - "Tak pozdrav přece, Hafie,"
            \item přímá řeč neznačená
            \item polopřímá řeč -  Juraj Hordubal potichoučku usedá na svůj kufřík. Moudrá je Polana a ví si rady; jak to tu vyzírá, hádal bys, že má dvanáct krav, dvanáct i víc
        \end{itemize}
        \item\textbf{jazykové prostředky a jejich funkce ve výňatku}
        \begin{itemize}
            \item spisovný jazyk
            \item cizí slova - gazdovat, jizba, gombík (slovenština)
            \item vyskloňovaná anglická slova uprostřed českých vět - v shaftu, teddy-bear, lowerdecku
            \item historismy - okov, vahadlo
            \item archaismy - roubení
            \item knižní - pravil
            \item ... aby lépe nastínil atmosféru vesnice
            \item v přímé řeči postav krátké věty
            \item ve vnitřním dialogu a vnímání Hordubala dlouhá souvětí - používá středníky
        \end{itemize}
        \item\textbf{tropy a figury a jejich funkce ve výňatku}
        \begin{itemize}
            \item metafora - peklo (ve smyslu horko), listy magazínů
            \item přirovnání - svítilo to jako sluníčko
        \end{itemize}
    \end{itemize}
    \section*{Literárně historický kontext a autor}
    Karel Čapek je jeden z nejznámějších a nejvýznamnějších českých spisovatelů (významný fejetonista), působil jako redaktor Národních listů a Lidových novin.
    Vytvořil novinový sloupek. Je představitelem demokratického proudu po WWI.
    Spoluzaložil spolek pátečníci - on, bratr, Peroutka, TGM, Beneš.
    Hordubal je první částí tzv. noetické trilogie (ještě Povětroň a Obyčejný život) - díla zaměřená na filozofický rozbor lidského vnímání pravdy. Trilogie je považována za jeden z jeho vrcholů.
    Mezi další významná díla patří např. RUR, Krakatit, Bílá nemoc, Válka s mloky, Hovory s TGM.
    Tvořil převážně v meziválečném období v 1. polovině 19. století.
    Spolutvořil i se svým bratrem Josefem - Pejsek a kočička, Dášeňka, neboli život štěněte, Ze života hmyzu.
    Obrovský úspěch jeho děl, sedmkrát nominován na Nobelovu cenu.
    Byly natočeny dva filmy o Hordubalovi. Další díla také zfilmována.
\end{document}