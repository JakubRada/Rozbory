\documentclass[11pt]{article}
\usepackage[a4paper, total={18cm, 27cm}]{geometry}
\usepackage[czech]{babel}
\usepackage[utf8]{inputenc}

\begin{document}
    \begin{center}
        \underline{\textbf{\Huge Vilém a Alois Mrštík - Maryša}}
    \end{center}
    \section*{Úryvek}
    MARYŠA (vyjde ze statku): Copak chcete?\\
    LÍZAL (probere se z myšlenek): Podé mně čagan - Tamhle leží -\\
    MARYŠA (podává mu hůl)\\
    LÍZAL: Kde je matka?\\
    MARYŠA: V kuchyni poklízí. Bude brzo poledně. (Vejde do zahrádky a trhá cosi.) Tatínku, co budeme dělat s tó turkyňó? Na poli ju žeró ptáci.\\
    LÍZAL: Nech včil turkyňu turkyňó a poď sem. Něco ti mám říct. - Kde seš?\\
    MARYŠA (vyjde ze zahrádky a zavírá za sebou branku)\\
    LÍZAL (bere ji za ruku): Co to máš?\\
    MARYŠA: Petržel. (V celém svém chování jeví bázeň a zlou předtuchu.)\\
    LÍZAL (pustí jí ruku): Jakpak, Maryško, což - (měkce) na vdavky nemáš ešče chuť?\\
    MARYŠA (utrhne se): Co by! (Rozhodně.) Já se ešče nechcu vdávat.\\
    LÍZAL: No mlč, máš už čas. Na svatýho Barnabáška už ti bude dvacet let.\\
    MARYŠA (hodí sebou a ustoupí ke dveřím): Eh, nechte marnéch řečí. (Má se k odchodu, najednou se obrátí ve dveřích.) A co za ženicha? Keho?\\
    LÍZAL (krotí se): No, ani bys neřekla. Hádej!\\
    MARYŠA (svěsí hlavu a přemýšlí)\\
    LÍZAL: Néni daleko odtáď - - -\\
    MARYŠA (vzhlédne po Lízálovi)\\
    LÍZAL: A má mlén.\\
    MARYŠA (vytáhne se v těle a zasvítí očima. - Zděšeně.) Snáď ne - - -\\
    LÍZAL (kývne hlavou radostně): Vávra je to - - -\\
    MARYŠA (vyjeveně a polozmateně): Vdovca? (Dá se do trpkého, ne bujného, ale dosud ještě bezstarostného, jasného smíchu.) Vdovca? Se třema dětima? (Zase se divoce směje.) No to ja! -\\
    LÍZAL (rozzlobeně sebou trhne): No co - vdovec nevdovec, ale je dobré, statečné muž -\\
    MARYŠA (zaraženě, zvážní uprostřed nejhlubšího smíchu, pustí kliku u dveří a temně): Snáď to nemyslíte doopravdy?\\
    LÍZAL (stále více pod sebe a k sobě víc než k Maryši): Co bych nemyslel?! Mně se líbí. Nepije, nekarbaní - chlapík jako dub - každá žena može se s něm dobře mět.\\
    MARYŠA (kysele): Enem že jé žádná nechce. Nebožku bil a děvečky si o něm povídajó všelijaký věci. Že se nestydí. - Děti má tři a ještě poméšlí na svobodno děvčicu - - -\\
    LÍZAL (vpadne): Co děti, děti! Šak só to hodný děti!\\
    MARYŠA: Kemu se líbí. Do hrobu se připravit nedám. Šla bych tam, myslím, jak nebožka. Šak ju na svědomí má enem on.\\
    LÍZAL (mlaská jazykem): Co - co - (Klepe holí do země.) Co ty víš! To só samý ženský plky. To se jen tak mluví.\\
    MARYŠA: Ať mluví, (vzdorně), ale Vávru já si nevezmu!\\
    LÍZAL (podrážděně): A Francka - teho žebráka bys chcela?\\
    MARYŠA: Šak sem neřekla - že chcu.\\
    LÍZAL: Ale chodí za tebó. - Myslíš, že to nevím? Myslíš, že su slepé a hluché? - Šak už temu bude konec. Ať ide žebrota, odkáď přišla. Ať si hledá žebrotu a neleze na půllán. Vávru si vemeš! (Krotí se k mírnosti.) Takové statek, takové mlén! Kam dáváš rozum, děvčico, kam's dala oči?\\
    MARYŠA (plačky a vzdorovitě): No, víte, děléte si, co chcete - - - ale já -\\
    LÍZAL (ji zakřikne): Mlč! (Bere hůl.) Tak budeš dělat, jak já poróčím. A včil si di!\\
    MARYŠA (chce něco říci, ale pak sebou trhne a bouchne za sebou dveřmi)\\
    LÍZAL (sám): Zpropadený děvčisko! (Vstává.) Ah! - Máma taky tak dělala a dnes je jako hodina.
    \newpage
    \section*{Analýza uměleckého textu}
    \begin{itemize}
        \item\textbf{zasazení výňatku do kontextu díla}
        \begin{itemize}
            \item ze začátku díla - První jednání, šestý výstup
            \item Maryša se dozvídá od Lízala (její otec), že si bude muset vzít Vávru
        \end{itemize}
        \item\textbf{téma a motiv}
        \begin{itemize}
            \item\textbf{téma:} nešťastné manželství
            \item\textbf{motiv:} nešťastná láska, manželství uzavřené pro peníze, život na vesnici, jed do kávy, krutost, poslušnost
        \end{itemize}
        \item\textbf{časoprostor}
        \begin{itemize}
            \item\textbf{čas:} konec 19. století
            \item\textbf{prostor:} prostředí moravské vesnice
        \end{itemize}
        \item\textbf{kompoziční výstavba}
        \begin{itemize}
            \item členěno podle vzoru antického dramatu do 5 jednání (expozice, kolize, krize, peripetie, katastrofa)
            \item první čtyři jednání rozdělena do 10-15 výstupů, poslední má pouze 6 (vyvrcholení děje, končící otrávením Vávry)
            \item chronologická kompozice (po 3. jednání je 2 roky skok, respektive 4. jednání je 2 roky po 1.)
        \end{itemize}
        \item\textbf{literární žánr a druh}
        \begin{itemize}
            \item\textbf{druh:} drama
            \item\textbf{žánr:} tragédie (vzor z antiky)
        \end{itemize}
        \item\textbf{vypravěč / lyrický subjekt}
        \begin{itemize}
            \item vypravěč chybí, děj je sdělován pomocí replik spojených převážně do dialogů mezi postavami
            \item du-forma
            \item autor vstupuje do děje pomocí scénických poznámek
        \end{itemize}
        \item\textbf{postava}
        \begin{itemize}
            \item\textbf{Maryša (hlavní postava) -} mladá, citlivá selská dívka, dcera Lízala. Miluje Francka, ale poslouchá své rodiče (náboženský vliv) a ti ji donutí vzít si Vávru. Prožívá vnitřní konflikt mezi vírou (poslechnout rodiče, zůstat s Vávrou) a touhou po opravdové lásce.
            \item\textbf{Lízal a Lízalka (vedlejší postavy) -} rodiče Maryši, majetní sedláci. Jde jim více o majetek než o jejich dceru. Přísní, krutí, lakotní. Otec je mírnější a nakonec si uvědomí svojí chybu.
            \item\textbf{Vávra (hlavní postava) -} bohatý mlynář, vdovec se třemi dětmi. Krutý, sobecký, agresivní, týrá Maryšu.
            \item\textbf{Francek (vedlejší postava) -} rekrut na vojnu, věrný, statečný, hrdý, buřič. Miluje Maryšu, ale musí odejít na vojnu, přemlouvá jí, aby šla s ním.
        \end{itemize}
        \item\textbf{typy promluv}
        \begin{itemize}
            \item většinou dialogy
        \end{itemize}
        \item\textbf{jazykové prostředky a jejich funkce ve výňatku}
        \begin{itemize}
            \item repliky hlavně nespisovně
            \begin{itemize}
                \item hanácké a slovácké nářečí - působí to věrněji, charakterizuje to prostředí a postavy
                \item \textit{Já se ešče nechcu vdávat.}; \textit{Podé mně čagan.}
            \end{itemize}
            \item scénické poznámky a úvody k jednáním psány spisovnou češtinou
            \item jednoduché věty nebo krátká souvětí
            \item časté pomlčky uprostřed vět - simulují pomlky v mluvené řeči
            \item autor přidává poznámky k replikám (v závorkách) - lepší vyznění emocí postav
            \item využití písní, hudby
            \item realistické popisy vesnického prostředí a obyvatel
        \end{itemize}
        \item\textbf{tropy a figury a jejich funkce ve výňatku}
        \begin{itemize}
            \item\textbf{figury}
            \begin{itemize}
                \item\textbf{gradace -} Vdovca? Vdovca? Se třema dětima? No to ja! - vyjádření rozzlobení nad otcovým návrhem
            \end{itemize}
            \item\textbf{tropy}
            \begin{itemize}
                \item\textbf{přirovnání -} chlapík jako dub, Máma taky tak dělala a dnes je jako hodina.
                \item\textbf{hyperbola -} Šla bych tam, myslím, jak nebožka. - zveličuje, jak hrozný Vávra je
            \end{itemize}
        \end{itemize}
    \end{itemize}
    \section*{Literárně historický kontext a autor}
    Maryša je dílem dvou bratrů, Aloise a Viléma, Mrštíkových.
    Narodili se na moravském venkově, což jim bylo inspirací ke všem jejich dílům.
    Oba tvořili i samostatně. Aloisovo nejznámější dílo je románová kronika Rok na vsi, kde popisuje život na moravské vesnici v průběhu 4 ročních období.
    Z Vilémových samostatných děl je nejvýznamnější Pohádka máje.
    Je také významným překladatelem, přeložil např. Dostojevského, Puškina nebo Turgeněva.
    Společně pak napsali ještě Bavlnkovy ženy a jiné povídky.
    Maryša je jejich vrcholné dílo a je to i vrchol českého realismu.

    Tvořili na přelomu 18. a 19. století.
    Psali hlavně realismus až naturalismus především ze života na vesnici.
    Vilém přispíval do časopisu Ruch a později i do Lumíru, redigoval Moravskoslezskou revue.
    Vilém byl propagátor ruského realismu (Dostojevský, Gogol) a jako kritik obhajoval francouzské naturalisty jako Zolu nebo Maupassanta.
    Maryša byla přijata sporně, kvůli svému nelichotivému zobrazení vztahů na vesnici, ale dodnes je jednou z nejpopulárnějších divadelních her.
    Na její motivy vznikla opera, balet i film. Je také zmíněna a inspirací ve hře Záskok z Divadla Járy Cimrmana.
    \section*{Zdroje}
    \begin{verbatim}
        Čtenářský deník nejen k maturitě, Jiří Mrákota - vydavatelství jazykové literatury
        Literatura v kostce pro SŠ, Marie Sochrová
        Český jazyk a literatura, Marie Sochrová
        https://cs.wikipedia.org/wiki/Maryša
        https://www.rozbor-dila.cz/marysa-rozbor-dila-k-maturite/
    \end{verbatim}
\end{document}