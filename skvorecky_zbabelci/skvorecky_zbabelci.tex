\documentclass[11pt]{article}
\usepackage[a4paper, total={18cm, 27cm}]{geometry}
\usepackage[czech]{babel}
\usepackage[utf8]{inputenc}

\begin{document}
    \begin{center}
        \underline{\textbf{\Huge Josef Škvorecký - Zbabělci}}
    \end{center}
    \section*{Úryvek}
    Z hlavní budovy se vyhrnuli pánové s modročervenýma
páskama. Každý měl v ruce lejstro. Jak vyšli do deště, začali si
ohrnovat límce a strkat lejstra do kapes. Potom se rozešli po dvoře.
Pan doktor Bohadlo, s tváří rozzářenou a důležitou, přišel k nám.
"Tak hoši, seřadíme se a pudem," řekl.
"Kam?" řekl Benno.
"Máme vnitřní město až k Poznerově továrně a pak dolů
kolem starého židovského hřbitova, přes ghetto a zpět k pivováru."
"A co budem dělat?"
"Vždycky tři hodiny budem konat tuhle obchůzku a pak
budem mít zas tři hodiny volno."
"To dem zrovna?"
"Ano. Teď je půl jedné, výjimečně budem hlídkovat jen dvě a
půl hodiny."
"A na voběd nepudem?"
"To vydržíte. Pro jednou," zasmál se pan doktor Bohadlo.
"To je sprostý. Bez žrádla nic nedělám," řekl Benno a myslel to
vážně. Ale pan doktor Bohadlo to vzal jako vtip.
"Dobrá. Bude to taková malá oběť pro vlast," řekl. "Teď jste na
vojně. Seřaďte se do dvojřadu za mnou."
Potom se k nám obrátil zády a zvedl ruku kolmo do výše jako
v Sokole, když se má dělat zákryt.
"Hovno," řekl potichu Benno.
Tři kluci, co jsem je neznal jménem, se horlivě seřadili za pana
doktora Bohadlo. Ten lichý se na nás otočil a Benno strčil Harýka k
němu. Harýk se postavil vedle něho a dal ruce do kapes gumáku.
Pan doktor Bohadlo stál pořád se zdviženou paží, s baňatými lýtky
a se zadnicí naditou v uzoučkých pumpkách. Dva kluci v první
dvojici za ním stáli v pozoru. Třetí kluk vedle Harýka taky. Harýk
stál s ohnutými zády a s ohrnutým límcem v dešti.
"Pochadémchod!" řekl pan doktor Bohadlo a mrsknul levou
nohou dopředu. Tři kluci se všichni najednou sklonili k levé straně
a vykročili levičkama. Harýk se opozdil, ale hned přidal do kroku.
šli jsme s Bennem za nima. Za chvíli už jsme byli pozadu.
"Sakra," řekl jsem.
"Krávy," řekl Benno a pak jsme zrychlili krok a připojili jsme se
k mašírující koloně. Mašírovali jsme s panem doktorem Bohadlem v
čele po dlážděné vozovce k bráně. Nebylo to vlastně tak
nepříjemné, jít stejným krokem. Rozhlédl jsem se a viděl jsem, že se
na nás nikdo nedívá. Tak to ani nebylo blbé. Ostatně, kolem nás
mašírovali samé takové stonožky s modročervenýma páskama v
čele. Jak jsme se blížili k bráně, viděl jsem, že je zavřená. Stál před ní
chlap v uniformě a s četařskými knoflíky na ramenou. Před sebou
měl opřenou pušku s bajonetem. Pan doktor Bohadlo šel rovnou k
bráně. Četař pozvedl pušku za hlaveň a otevřel bránu před námi.
Prošli jsme jako londýnská garda a pan doktor Bohadlo bezvadně
zahnul vpravo. Mašírovali jsme po silnici k mostu. Ohlédl jsem se
vlevo k Port Arthuru. Potom jsme šli přes most a já se díval na
Irenino okno. Nikde nic. Mašírovali jsme tiše k nádraží. Na rohu
okresního domu stála paní Mánesová. Jak nás spatřila, sebrala se a
připojila se k nám.
    \section*{Analýza uměleckého textu}
    \begin{itemize}
        \item\textbf{zasazení výňatku do kontextu díla}
        \begin{itemize}
            \item před tím byl Danny zajat Němci a mohl být zastřelen, ale doktor vyjednal s Němci jejich propuštění
            \item první polovina, těsně po zahájení odporu proti okupaci, chlapci se přihlásili do odboje v pivováru a jdou na obchůzku po městě
            \item po chvíli je to přestane bavit a utečou z pivováru
        \end{itemize}
        \item\textbf{téma a motiv}
        \begin{itemize}
            \item\textbf{téma: }život mladých v maloměstě a jejich konfrontace s revolucí (přijde jim to jenom jako zábava než doopravdy nepřijedou tanky, pak jsou bojovat)
            \item\textbf{motivy: }jazz, láska, kamarádství, předstírání aktivity
        \end{itemize}
        \item\textbf{časoprostor}
        \begin{itemize}
            \item\textbf{čas: }týden v květnu 1945 - autorova současnost, doba revoluce proti německé okupaci
            \item\textbf{prostor: }reálný, konkrétní - české maloměsto - Kostelec, autor chtěl zobrazit, jak probíhala revoluce na maloměstě (kontrast s Prahou)
        \end{itemize}
        \item\textbf{kompoziční výstavba}
        \begin{itemize}
            \item na začátku je věnování (Zdence, té holce, Co jsem potkal v Praze)
            \item a také dva citáty - Romain Rolland (Každé dílo, které trvá, je vytvořeno ze samé podstaty své doby; umělec je sám nevybudoval; je v něm vypsáno, co vytrpěli, milovali, o čem snili jeho druhové.) a Ernest Hemingway (Spisovatelovým řemeslem je mluvit pravdu.)
            \item dílo rozděleno do 7 částí - každá část odpovídá jednomu dni příběhu
            \item části jsou uvozeny jako deník (např. Pátek 11.V.1945)
            \item chronologická výstavba
        \end{itemize}
        \item\textbf{literární žánr a druh}
        \begin{itemize}
            \item\textbf{druh: }epika, próza
            \item\textbf{žánr: }román (často označován za první román deziluze)
        \end{itemize}
        \item\textbf{vypravěč / lyrický subjekt}
        \begin{itemize}
            \item osobní vypravěč - postava Dannyho Smiřického
            \item nespolehlivý vypravěč - říká nám informace ze svého pohledu, nevíme vše a jestli je to pravda
            \item ich-forma
        \end{itemize}
        \item\textbf{postava}
        \begin{itemize}
            \item Danny Smiřický - hlavní postava, fiktivní, přirozená, částečně autobiografická (autor tvrdí, že ne, ale jsou si podobní - Škvoreckého ideální já), neutrální - pracuje v továrně a ve volném čase hraje jazz s kamarády, působí cynicky, domýšlivě, ale maskuje tím svou vnímavost a citlivost, kritizuje pokrytectví maloměšťáků; je zamilovaný do Ireny, ale pořád flirtuje s ostatními dívkami, když přijedou německé tanky, pomůže jeden sestřelit kulometem
        \end{itemize}
        \item\textbf{vyprávěcí způsoby}
        \begin{itemize}
            \item ich-forma, přirozené dialogy a dlouhé Dannyho monology
        \end{itemize}
        \item\textbf{typy promluv}
        \begin{itemize}
            \item (v ukázce) hlavně přímá řeč - přirozené dialogy
        \end{itemize}
        \item\textbf{jazykové prostředky a jejich funkce ve výňatku}
        \begin{itemize}
            \item ve vypravěčových pasážích spisovná čeština
            \item v dialozích nespisovný jazyk - působí to autentičtěji, živěji, jak spolu mluví mladí lidé
            \begin{itemize}
                \item obecná čština - dem, voběd, nepudem
                \item vulgarismy - hovno
                \item expresivní slova - žrádlo
                \item slang
                \item anglické a německé výrazy
                \item cizí slova převedená do češtiny, aby se vyslovovala stejně - ólrajt, šatap
            \end{itemize}
            \item v promluvách krátké úderné věty - akční dialog
            \item vypravěč a jeho vnitřní monolog - delší věty a souvětí
        \end{itemize}
        \item\textbf{tropy a figury a jejich funkce ve výňatku}
        \begin{itemize}
            \item metafora - stonožky s modročervenýma páskama
        \end{itemize}
    \end{itemize}
    \section*{Literárně historický kontext a autor}
    Josef Škvorecký byl významný spisovatel a překladatel hlavně 2. poloviny 20. století.
    Považován za prvního představitele románu deziluze - zobrazoval revoluci takovou, jaká ve skutečnosti byla.
    Zbabělce napsal už 1948, ale věděl, že by dílo vadilo, vydal ho až o 10 let později, ale stejně přišel o místo.
    V roce 1968 emigroval do Kanady a založil exilové nakladatelství 68 Publishers - vydával v Československu zakázaná díla.
    Danny není autobiografická postava, ale mají mnoho podobného - je to jeho ideální já, vystupuje i v dalších románech - Tankový prapor, Příběh inženýra lidských duší.
    Jiná díla jsou Sedmiramenný svícen (židovská tematika), detektivní trilogie poručíka Borůvky. Překládal řadu autorů, např. Hemingwaye - podobný styl psaní.
    Manželka Zdena Salivarová také spisovatelka.
    Dílo přijato dobře a dodnes jsou oblíbená. Několikrát byl napsán scénář k filmu, ale samotný snímek nikdy nevznikl.
    Ve stejné době tvořil např. Kundera, Vaculík, Klíma, Bohumil Hrabal. Ve světě Heller, Moravia, beat generation.
\end{document}