\documentclass[11pt]{article}
\usepackage[a4paper, total={18cm, 27cm}]{geometry}
\usepackage[czech]{babel}
\usepackage[utf8]{inputenc}
\usepackage{multicol}

\begin{document}
    \begin{center}
        \underline{\textbf{\Huge William Shakespeare - Hamlet}}
    \end{center}
    \section*{Úryvek}
    \begin{multicols}{2}
    Ham. 

    Být čili nebýt, –ta jest otázka: –

    víc důstojno-li ducha trpěti

    od střel a praků zlého osudu,

    neb ozbrojit se proti moři běd

    a ukončit je vzpourou.–Umřít –spát –

    nic víc; –a spánkem, řekněm, –ukončit

    bol srdce, tisíc přirozených ran,

    jichž tělo dědicem, –toť skonání,

    jak si ho vroucně přáti. –Umřít –spát;

    spát, –snad že snít! –ah –tady vázne to: –

    neb jaké sny as mohou přijíti

    v tom spánku smrti, když jsme setřásli

    svá pouta smrtelná, –v tom váháme;

    toť ohled, který daří neštěstí

    tak dlouhým životem; neb, kdož by chtěl

    nést bičování dob a výsměšky,

    kdo útisk mocných, pyšných pohrdu,

    hlod lásky zhrzené, práv průtahy,

    a řádu svévoli a ústrky,

    jež snáší trpělivá zásluha

    od nehodných, když sám si může dát

    mír pouhou jehlou? –Kdo by nésti chtěl

    ta břemena a stenal, potil se

    pod tíhou žití, leda že jen strach

    před něčím po smrti, kraj neznámý,

    od jehož břehů žádný poutník již

    se nevrací, nám vůli opoutá

    a nutká nás spíš nést zla přítomná

    než prchat k jiným, o nichž nevíme?

    Tak skety činí svědomí z nás všech

    a přirozenou barvu odvahy

    zchorobní bledý nátěr myšlenky,

    a veliké a vážné podniky

    tím ohledem od směru odbočí

    a ztratí jméno skutků. –Ticho teď!

    Aj, hle, toťspanilá Ofelie!

    V svých modlitbách, ó Nymfo, veškerých

    mých hříchů vzpomínej!

    Ofel.

    Můj vzácný pane,jakjste se měl ten celý dlouhý čas?

    Ham.

    Mé slušné díky; –dobře, dobře, dobře.

    Ofel.

    Můj princi, zde mám od vás památky, jež dávno jsem si přála

    navrátit; teď, prosím, přijměte je.

    Ham.

    Ne, –’ já ne;já nikdy jsem vám nedal ničeho.

    Ofel.

    Můj vzácný pane, víte, že jste dal,

    a k tomu slova sladkodechá tak,

    že činila ty věci dražšími.

    Kdyžnyní svojí vůně pozbyly,

    je vezměte; neb duši šlechetné

    se chudý stanou dary bohaté,

    když dárce ukáže se nevlídným.

    Zde, pane můj.

    Ham.

    Ha, ha! Jste počestná?

    Ofel.

    Můj princi?

    Ham.

    Jste krásná?

    Ofel.

    Co vaše Milost myslí tím?

    Ham.

    že, jste-li počestná a krásná, vaše počestnost neměla by se

    pouštět do hovoru s vaší krásou.

    Ofel.

    Může-li krása, můj princi, míti lepší družku než počestnost?

    Ham.

    Ó zajisté; neboťmoc krásy přetvoří spíše počestnost z toho,

    čím jest, na kuplířku, než počestnost přeměníkrásu na svou podobu. To

    bývalo druhdy protismyslem, ale teď to potvrzuje čas. Já vás kdysi miloval.

    Ofel.

    Můj princi, vskutku, choval jste se tak, že věřila jsem tomu.

    Ham.

    To jste mi věřit neměla; neboť ctnost nedá se vštěpovati tak na

    náš starý peň, abychom nezachovali starou příchuť. Já vás nemiloval.

    Ofel.

    Tím více byla jsem klamána.

    Ham.

    Jdi do kláštera; proč chtělabys být roditelkou hříšníků?

    Já sám jsem jakž takž poctivý a přec mohl bych se vinit z takových

    věcí, že bylo by lépe, aby mne matka byla neporadila.Jsem

    velmi hrdý, mstivý, ctižádostivý; mám více nepravostí k

    službám, než myšlenek je obsáhnout, než obraznosti dáti jim

    tvar, než času vykonati je. Nač se mají takoví chlapi jako já

    plaziti mezi nebem a zemí? Jsme všichni arcilotři; nevěř z

    nás nikomu. Jdi po svých, do kláštera. –Kde je váš otec?

    Ofel.

    Doma, můj princi.

    Ham.

    Dejte za ním dveře uzamknout, aby jinde ze sebe blázna nedělal,

    než ve svém vlastním domě. S bohem!

    Ofel.

    Ó, pomozte mu, dobra nebesa!

    Ham.

    Vdáš-li se, dám ti tuto kletbu do výbavy: buď cudná jako led,

    čistá jak sníh, pomluvě neujdeš. Jdi do kláštera, jdi; s bohem.

    A –jestliže se mermomocí vdáti chceš, vezmi si blázna; neboť

    lidé moudří vědí velmi dobře, jaké vy z nich naděláte

    netvory. Jdi do kláštera; a hezky si pospěš. –S bohem!

    Ofel.

    Ó, uzdravte jej, nebes mocnosti!

    Ham.

    O vašem se malování slyšel jsem též, a dosti mnoho. Bůh vám

    dal jeden obličej a vy si děláte jiný, vy točíte se, hopkujete,

    šepotáte, přezdíváte božím tvorům a, při své chtivosti stavíte

    se nevědomými. Jděte mi! Mám toho po krk; zbláznilo mne

    to. Já pravím: nechceme žádných sňatků víc! Ti? kdo již jsou

    ženati, ať zbudou na živu, –ažna jednoho! –ti ostatní ať

    zůstanou, jak jsou. Do kláštera, –jdi. (Odejde.)

    Ofel.

    Ó, jaký vznešený tu zmařen duch! –

    zrak dvořanův, hlas učencův, meč reka,

    růže a naděj říše kvetoucí,

    všech řádů kadlub, mravů zrcadlo,

    vzor vzoru dbalých zničen nadobro!

    A já, všech žen ta nejnešťastnější

    a nejbídnější, kteráž ssála jsem

    med jeho slibů hudbou zvučících,

    teď zřím ten rozum čacký, vševládný

    jak popukané zvonky líbezné,

    vše rozladěně, drsně znějící;

    tu nevyrovnatelnou postavu

    a kvetoucího mládí obličej,

    vše na zmar přivedeno šíleností! –

    Ó běda mi, že shlédla zrakem svým jsem

    to,co shlédla jsem, a zřím, co zřím!
    \end{multicols}
    \section*{Analýza uměleckého textu}
    \begin{itemize}
        \item\textbf{zasazení výňatku do kontextu díla}
        \begin{itemize}
            \item přibližně v polovině díla - 3. jednání, před usmrcením Polonia, rozjímá nad sebevraždou a nespravedlnosti v životě
        \end{itemize}
        \item\textbf{téma a motiv}
        \begin{itemize}
            \item\textbf{téma: }mravní dilema - touha splnit ideály x pocit potřeby potrestat zlo; smysl života; osud
            \item\textbf{motivy: }vražda, láska, pomsta, nespravedlnost, šílenství
        \end{itemize}
        \item\textbf{časoprostor}
        \begin{itemize}
            \item\textbf{čas: }začátek 17. století - nehraje to roli, autorova současnost
            \item\textbf{prostor: }hrad Elsinor v Dánsku - reálný svět, konkrétní místo
        \end{itemize}
        \item\textbf{kompoziční výstavba}
        \begin{itemize}
            \item rozděleno do 5 jednání - má to strukturu jako antické drama (expozice, kolize, krize, peripetie, katastrofa); jednotlivá jednání se dělí na 2-7 scén, ty obsahují repliky
            \item děj chronologický, přehledný
        \end{itemize}
        \item\textbf{literární žánr a druh}
        \begin{itemize}
            \item\textbf{druh: }drama
            \item\textbf{žánr: }tragedie
        \end{itemize}
        \item\textbf{vypravěč / lyrický subjekt}
        \begin{itemize}
            \item vypravěč není, je to drama
        \end{itemize}
        \item\textbf{postava}
        \begin{itemize}
            \item Hamlet - hlavní, fiktivní, přirozená - kralevic, syn zesnulého krále, po zjištění, že krále zabil jeho strýc Claudius, který si vzal jeho matku a je nyní králem, prožívá vnitřní dilema, protože chce zachovat ideály, ale zároveň chce otce pomstít; ve svých promluvách uvažuje o smyslu života
            \item Claudius - hlavní, fiktivní, přirozená, negativní - nový král po vraždě Hamletova otce, zrádný, zákeřný, ctižádostivý, vezme si Hamletovu matku
            \item Polonius - vedlejší, fiktivní, přirozená - komoří, velmi oddaný králi, otec Ofélie, snaží se podlézat králi - odposlouchává Hamleta za závěsem, ten ho omylem zabije
        \end{itemize}
        \item\textbf{vyprávěcí způsoby}
        \begin{itemize}
            \item du-forma
        \end{itemize}
        \item\textbf{typy promluv}
        \begin{itemize}
            \item dialogy mezi postavami i rozsáhlé monology o smyslu života (hlavně Hamlet)
        \end{itemize}
        \item\textbf{jazykové prostředky a jejich funkce ve výňatku}
        \begin{itemize}
            \item spisovný zastaralý jazyk
            \item dnes už knižní slova - neb, bol, druhdy, slovesa končící na -ti, -li, čacký, arcilotři, zřím
            \item často dlouhá souvětí, nepřirozený slovosled
            \item zní to vznešeněji
        \end{itemize}
        \item\textbf{tropy a figury a jejich funkce ve výňatku}
        \begin{itemize}
            \item často inverze - nepřirozený slovosled
            \item metafora - tíha žití
            \item přirovnání - vševládný jak popukané zvonky líbezné
        \end{itemize}
    \end{itemize}
    \section*{Literárně historický kontext a autor}
    William Shakespeare je asi nejznámějším dramatikem. Stal se hercem a pak majitelem herecké společnosti.
    Byl oblíbený u královské rodiny.
    Psal historické hry (Richard III., Jindřich IV., Král Jan), tragedie (Romeo a Julie, Othello, Macbeth, Julius Caesar) a komedie (Sen noci svatojánské, Zkrocení zlé ženy).
    Byl i básníkem a napsal 154 sonetů.
    Nejprve psal historické hry, pak Romea a Julii, pak Hamleta.
    Příběh odvozen od legengy Amleth ze 13. století.
    Tvořil na přelomu 16. a 17. století (Renesance - návrat k antice, zásady dramatu, rozvoj člověka duševně i fyzicky).
    Jeho současníky jsou Lope de Vega (Fuente Ovejuna), Cervantes (Don Quijote de la Mancha).
    Díla jsou oblíbená do dnešní doby, mnoho bylo zfilmováno.
\end{document}