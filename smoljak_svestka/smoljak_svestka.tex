\documentclass[11pt]{article}
\usepackage[a4paper, total={18cm, 27cm}]{geometry}
\usepackage[czech]{babel}
\usepackage[utf8]{inputenc}

\begin{document}
    \begin{center}
        \underline{\textbf{\Huge Ladislav Smoljak, Zeněk Svěrák - Švestka}}
    \end{center}
    \section*{Úryvek}
    PULEC: V první řadě by to chtělo žebřík.
    \\MOTYČKA: Dáma podává ruku vždycky jako první.
    \\HÁJEK: Tak to bude kámen úrazu, Sváťo. Je tady totiž disponent.
    \\MOTYČKA: Pokud ovšem před ní nestojí jiná dáma. Starší dáma podává ruku mladší dámě.
    \\PULEC: Co jsi říkal?
    \\MOTYČKA: Že starší dáma...
    \\PULEC: Ale ty ne!
    \\HÁJEK: Že je tady disponent.
    \\PULEC: TO znamená co?
    \\(V té chvíli vyjde z vechtrovny Patka a táhne za sebou dlouhou telefonní pásku, jejíž druhý konec je stále ještě v přístroji.)
    \\PATKA: To znamená, že záleží na mně, komu žebřík půjčím a komu ne. Právě jsem se na to ptal kolegy ze sousední stanice a jasně mi tady sděluje, že bych byl blázen, kdybych vám ho půjčil.
    \\HÁJEK: Ukažte... (Vezme mu z ruky pásku a prohlíží si zprávu.) Kdo to psal?
    \\PATKA: Výpravčí Tumpach.
    \\HÁJEK: Tumpach! Tak on ještě slouží, dědek stará... To je ale rukopis... rozklepaný je to, rovnou čárku neudělá... (Čte a ručkuje po pásce až k telegrafnímu přístroji ve vechtrovně.) Milý Kamile, předem mého telegramu přijmi srdečný pozdrav ze sousední stanice. Jinýmu bych žebřík půjčil, ale Hájkovi naser. (Vyběhne dotčeně z vechtrovny.)
    \\HÁJEK: To vy si dovolujete s Tumpachem takovéhle věci služebním telegrafem? Drážní telegraf používáme výhradně ke služebnímu styku. Jakákoli soukromá sdělení jsou nepřípustná.
    \\PATKA: To není žádné soukromé sdělení. To je vyžádaná služební informace stran drážního žebříku.
    \\MOTYČKA: Pán smí podat ruku dámě jako první jedině v případě, pokud tato je tak mladá, že by mohla býti jeho dcerou.
    \\PULEC: Hele, vykašlem se na jeho žebřík a hodíme tam lano.
    \\MOTYČKA: Slušný chování, to je věda.
    \\HÁJEK: Na co lano?
    \\MOTYČKA: O tom jsou celý knihy.
    \\PULEC: Na větev. Přehodíme lano přes větev a přivážeme na něj košík.
    \\HÁJEK: Počkej, Sváťo, lano, košík, ty zas vidíš jen ty své horolezecké věci. My tady řešíme úplně jiné problémy.
    \\PULEC: Jaký problémy?
    \\HÁJEK: Že nemáme žebřík.
    \\PULEC: Vždyť o tom mluvím.
    \\HÁJEK: Ty mluvíš o nějakém laně a koši. Já mluvím o žebříku.
    \\PULEC: Ale to spolu přece souvisí.
    \\HÁJEK: Jak to spolu souvisí?
    \\PULEC: Ježíši, vždyť přece víš, kvůli čemu jsme sem přijeli. Já s košem, ty s ruksakem, Motyčka s pytlem.
    \\MOTYČKA: A je to venku! Říkal jsem vám, Kamile, přijede Přemek a dovíte se to. Tak, Přemku, řekni mu, na co mám ten pytel.
    \\HÁJEK: Ty máš pytel?
    \\MOTYČKA: No jo, podívej. (Ukáže prázdný pytel.)
    \\HÁJEK: A na co ho máš?
    \\PULEC: Přemku, vzpamatuj se. Říkal jsi, že se rozdělíme na tři díly.
    \\HÁJEK: Co na tři díly?
    \\PULEC: No švestky přece!
    \\HÁJEK: (Pohlédne do koruny stromu a konečně si vzpomene.) Díky, Sváťo. Já jak toho mám víc, tak zapomenu jedno pro druhý. Díky!
    \section*{Analýza uměleckého textu}
    \begin{itemize}
        \item\textbf{zasazení výňatku do kontextu díla}
        \begin{itemize}
            \item ve druhé polovině díla, přibližně v polovině hry - před tím přijel Standa Pulec pomoct očesat švestku
        \end{itemize}
        \item\textbf{téma a motiv}
        \begin{itemize}
            \item\textbf{téma: }stáří (neschopnost myšlenku udržet a opustit), mystifikace diváka (vymyšlenou postavou nedoceněného českého génia)
            \item\textbf{motivy: }vědecká přednáška, ochotnické divadlo, železnice, situační a jazyková komika
        \end{itemize}
        \item\textbf{časoprostor}
        \begin{itemize}
            \item\textbf{čas: }blíže neurčený, nejspíše začátek 20. století před WWI - současnost údajného autora Járy Cimrmana
            \item\textbf{prostor: }reálný svět, vymyšlené místo - železniční stanice Středoplky - Jára Cimrman tuto hru napsal když byl v této stanici u místního vechtra
        \end{itemize}
        \item\textbf{kompoziční výstavba}
        \begin{itemize}
            \item dvě části
            \begin{itemize}
                \item seminář - herci mají vědecké referáty rozebírající oblast Cimrmanova působení - tentokrát zubařina a železnice, na konci semináře je písnička Šel nádražák na mlíčí
                \item hra - herci hrají Cimrmanovu jednoaktovou hru Švestka - působí to jako ochotnické divadlo, herci přehrávají - kontrast s vědeckým přístupem k referátům
                \item hra má podtitulek Jevištní sklerotikon
            \end{itemize}
            \item hra je chronologická
        \end{itemize}
        \item\textbf{literární žánr a druh}
        \begin{itemize}
            \item\textbf{druh: }drama
            \item\textbf{žánr: }komedie, jednoaktovka
        \end{itemize}
        \item\textbf{vypravěč / lyrický subjekt}
        \begin{itemize}
            \item není, je to drama, jen scénické poznámky
        \end{itemize}
        \item\textbf{postava}
        \begin{itemize}
            \item Hájek - hlavní, fiktivní, přirozená - bývalý vechtr, starý, ukazuje jeden z neduhů stáří - neschopnost udržet myšlenku (neustále zapomíná proč tam je), jeho slabost je potřesení rukou, že ji zapomíná pustit
            \item Motyčka - hlavní, fiktivní, přirozená - bratranec hájka, starý, ukazuje druhý neduh stáří - neschopnost myšlenku opustit (pořád mluví o tom samém, i když ho ostatní přerušují, vrací se k tomu)
            \item Sváťa Pulec - hlavní, fiktivní, přirozená - bývalý slavný český horolezec, který 7 krát pokořil Milešovku, ale poosmé spadl a je na vozíku, přijel pomoc očesat švestku, ale nedaří se mu přesvědčit ostatní, aby se do toho pustili (Hájek netuší, Motyčka si mluví svoje)
            \item Kamil - hlavní, fiktivní, přirozená - nový vechtr, mladý, odmítá půjčit žebřík, protože si chce švestku očesat sám, čeká na nějakou dívku až přijede vlakem
        \end{itemize}
        \item\textbf{vyprávěcí způsoby}
        \begin{itemize}
            \item du-forma
        \end{itemize}
        \item\textbf{typy promluv}
        \begin{itemize}
            \item repliky, dialogy, monology (Kamil na začátku)
        \end{itemize}
        \item\textbf{jazykové prostředky a jejich funkce ve výňatku}
        \begin{itemize}
            \item spisovný jazyk
            \item archaismy - vechtr, vechtrovna, šraňky
            \item historismy - telegraf
            \item v každé hře mají jeden vulgarismus - vtipné, ale ne nevhodné (Jinému bych žebřík půjčil, ale Hájkovi naser)
            \item občas obecná čeština - slušný chování, celý knihy (nejvíce Motyčka)
            \item jazyková komika - výpravčí Tumpach, Kryštof nastoupil je harant a bydlí v Polžicích nebo Bezdružicích, Motyčka si myslí, že Pulcův otec se jmenuje Žába
            \item spíše krátké věty v dialozích
        \end{itemize}
        \item\textbf{tropy a figury a jejich funkce ve výňatku}
        \begin{itemize}
            \item kámen úrazu - metafora
            \item vykašlem se na jeho žebřík - metafora
            \item koruna stromu - metafora
        \end{itemize}
    \end{itemize}
    \section*{Literárně historický kontext a autor}
    Zdeněk Svěrák a Ladislav Smoljak je známá literární dvojice, jsou hlavními představiteli Divadla Járy Cimrmana.
    Svěrák je učitelem češtiny, což mu značně pomohlo ve vytváření jazykové komiky ve hrách.
    Divadlo vyznačující se mystifikačním humorem, vytvořili postavu Járy Cimrmana, což je nedoceněný génius, který zasáhl do všech odvětví, a oni odkrývají jeho pozůstalost ve formě divadelních her, písní, \dots
    Zábavné pro všechny vrstvy lidí - jak jednoduchý, tak i hluboký humor; hlavně situační a jazyková komika, málo dobových poznámek.
    Inspirace Voskovcem a Werichem a jejich dialogovým humorem.
    Celkem vytvořili 15 her, 10 před revolucí. Švestka je až porevoluční. Hry parodují divadelní žánry a postupy.
    Další hry: Záskok, Akt, České nebe, Hospoda na mýtince, \dots
    Tvořili v 2. polovině 20. století a na začátku 21. století.
    Obzvláště po revoluci obrovský úspěch, je pořád vyprodáno.
    Společně jsou autory velmi známých českých filmů - Marečku, podejte mi pero; Na samotě u lesa, \dots
\end{document}