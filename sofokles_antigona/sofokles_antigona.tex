\documentclass[11pt]{article}
\usepackage[a4paper, total={18cm, 27cm}]{geometry}
\usepackage[czech]{babel}
\usepackage[utf8]{inputenc}
\usepackage[parfill]{parskip}

\begin{document}
    \begin{center}
        \underline{\textbf{\Huge Sofokles - Antigona}}
    \end{center}
    \section*{Úryvek}
    \textbf{Antigona}

    Chceš něco víc, než uchopě mne usmrtit?

    \textbf{Kreon}

    Ba více nic; neb to mně stačí úplně.

    \textbf{Antigona}

    Co tedy váháš? Mně z tvých řečí nelíbí

    se nic i nechci, by se mi kdy líbilo,

    a tak i tobě jednání mé protivno.

    Však čím bych slávy byla došla skvělejší,

    než ve hrob ukládajíc bratra vlastního?

    Ti všickni zde by vyřknuli, že schvalují

    ten skutek, kdyby jazyk nevázal jim strach.

    Však samovláda jest i jinak blažena,

    i mluvit jí a jednat možno jakkoliv.

    \textbf{Kreon}

    Ty jediná to vidíš, ne ti Kadmejští.

    \textbf{Antigona}

    I ti to vidí, avšak mlčí před tebou.

    \textbf{Kreon}

    A ty se jinak chovat než ti nestydíš?

    \textbf{Antigona}

    Vždyť pokrevence ctíti není mrzká věc.

    \textbf{Kreon}

    Což ten, jenž oním zhynul, není bratr tvůj?

    \textbf{Antigona}

    Můj bratr jest, z též matky, z otce jednoho.

    \textbf{Kreon}

    Proč tedy vzdáváš tomuto čest bezbožnou?

    \textbf{Antigona}

    V tom nepřisvědčí tobě padlý nebožtík.

    \textbf{Kreon}

    Ba bezbožnou, když ctíš jej rovně s bezbožným.

    \textbf{Antigona}

    Vždyť nezahynul otrok, nýbrž bratr můj.

    \textbf{Kreon}

    Leč hubil tuto zemi, druhý hájil jí.
    \section*{Analýza uměleckého textu}
    \begin{itemize}
        \item\textbf{zasazení výňatku do kontextu díla}
        \begin{itemize}
            \item přibližně v polovině knihy, poté co hlídači Polyneikova těla chytnou Antigonu, jak ho pohřbívá, a před tím než je Antigona zavřena v jeskyni na smrt
        \end{itemize}
        \item\textbf{téma a motiv}
        \begin{itemize}
            \item \textbf{téma:} vítězství morálky
            \item \textbf{motivy:} konflikt generací, krutý trest, předurčení
        \end{itemize}
        \item\textbf{časoprostor}
        \begin{itemize}
            \item před královským palácem v řeckých Thébách, v mytologických časech, kdy žila Antigona, dcera Oidipa
        \end{itemize}
        \item\textbf{kompoziční výstavba}
        \begin{itemize}
            \item dodržuje Aristotelovu zásadu 3 jednot - jedna dějová linie, jedno místo, krátký časový úsek
            \item chronologická
            \item jednoaktová hra - dělení do 14 výstupů
            \begin{itemize}
                \item střídá se výstup postav děje - episodion a výstup sboru - stasimon
            \end{itemize}
        \end{itemize}
        \item\textbf{literární žánr a druh}
        \begin{itemize}
            \item \textbf{žánr:} tragedie
            \item \textbf{druh:} drama
        \end{itemize}
        \item\textbf{vypravěč / lyrický subjekt}
        \begin{itemize}
            \item vypravěč není - je to drama
        \end{itemize}
        \item\textbf{postava}
        \begin{itemize}
            \item \textbf{heroizace člověka - }člověk zápasí s nepřízní osudu
            \item \textbf{generační konflikt - }mládí (rozum, mravy) x stáří (moc, autorita, zaběhlá dogmata)
            \item \textbf{Antigona:} jedna ze dvou dcer thébského krále Oidipa, i přes zákaz pohřbí svého bratra Polyneika, je za to svým strýcem Kreontem odsouzena ke smrti zavřená v jeskyni, oběsí se tam. Odvážná, má silné morální zásady, obětuje sama sebe pro dobrou věc.
            \item \textbf{Ismena:} Sestra Antigony, odmítne jí pomoct pohřbít bratra, ale po jejím odhalení chce na sebe vzít část viny a být zabita s Antigonou. Bojácná, ale věrná své sestře.
            \item \textbf{Kreon:} thébský král, dodrží své slovo, ale posléze toho lituje, když zabije Antigonu, jeho syn spáchá sebevraždu, stejně jako jeho manželka - přijde o všechno a lituje toho, krutý vladař, nebral v ohledu zvyklosti a proti vůli občanů prosadil svá rozhodnutí
        \end{itemize}
        \item\textbf{vyprávěcí způsoby}
        \begin{itemize}
            \item du-forma
            \item monology i dialogy
        \end{itemize}
        \item\textbf{typy promluv}
        \begin{itemize}
            \item drama - celé je to jakoby přímá řeč - repliky
        \end{itemize}
        \item\textbf{veršová výstavba}
        \begin{itemize}
            \item není
        \end{itemize}
        \item\textbf{jazykové prostředky a jejich funkce ve výňatku}
        \begin{itemize}
            \item spisovný jazyk
            \item zastaralý, archaismy a knižní slova
            \item \textbf{archaismy: }\textit{uchopě, pokrevenec, všickni}
            \item \textbf{knižní slova: }\textit{jest, oním, nebožtík, leč}
            \item \textbf{přechodníky: }\textit{ukládajíc}
            \item autentické znění z té doby
            \item přehozený slovosled, dlouhá souvětí o mnoho větách - zní to vznešeněji
        \end{itemize}
        \item\textbf{tropy a figury a jejich funkce ve výňatku}
        \begin{itemize}
            \item není zde moc tropů - lépe to simuluje skutečný dialog, který není moc bohatý
            \item figury také ne
        \end{itemize}
    \end{itemize}
    \section*{Literárně historický kontext a autor}
    Sofokles byl jedním z předních antických dramatiků.
    Hry nejenom psal, ale byl i hercem.
    Zavedl postavu třetího herce a omezil úlohu chóru (zpívaný dialog nebo hudební vstup do představení).
    Napsal více než 100 her, ze kterých se zachovalo pouze 7. Z nich nejznámější je Thébská trilogie (král Oidipus, Oidipus na Kolóně, Antigona), dále Elektra.
    Dalšími autory json Aischylos (Oresteia), Euripidés (Médea), Aristofanés.
    \end{document}