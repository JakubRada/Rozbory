\documentclass[11pt]{article}
\usepackage[a4paper, total={18cm, 27cm}]{geometry}
\usepackage[czech]{babel}
\usepackage[utf8]{inputenc}

\begin{document}
    \begin{center}
        \underline{\textbf{\Huge Ladislav Fuks - Spalovač mrtvol}}
    \end{center}
    \section*{Úryvek}
    K večeru, když bylo jídlo hotovo, vybídl pan Kopfrkingl Lakmé, aby si vzala tmavé hedvábné sváteční šaty s bílým krajkovým límečkem, a když si je oblékla, dovedl ji do jídelny, posadil za stůl, přinesl chlebíčky, mandle, kávu a čaj, otevřel rádio a pak si sedl ke stolu i on.
"Slyšíš, nebeská," usmál se něžně, "to, co teď právě hrají, je sbor a bas z Donizettiho Lucie z Lammermooru. Je to zajímavé. Je to tak dokonale pohřební hudba, a přece se u nás tak málo hraje. Kdo by si ji u nás dal zahrát v síni, měl by pohřeb vskutku vzácný. Paní Strunné kdysi hráli Nedokončenou, slečně Čárské Dvořákovo Largo a slečně Vomáčkové nedávno píseň o Poslední růži Friedricha von Flotowa. Vada je v tom, že hudbu si pro sebe zpravidla nevolí nebožtíci, ale že jim ji vyberou pozůstalí! A ti nevyberou podle vkusu mrtvých, ale podle vkusu svého. Ti nevyberou to, co se líbí jejich nebohým, ale to, co se líbí jim." Pak řekl: "Tohle je velká árie Lucie z třetího dějství. Zpívá ji nějaká výtečná Italka." A zatímco jedli a z rádia zněla árie Lucie, pan Kopfrkingl řekl: "Máme nejčistší život před sebou. Máme, nadoblačná, otevřený svět. Je nám otevřeno nebe," ukázal a pohlédl na strop, jako by upozorňoval na hvězdy, nádherný obraz nebo zjevení, "nebe, na kterém za celých těch devatenáct let, co jsme spolu, nepřelétl jeden jediný mráček, nebe, jaké někdy vidám nad mým Chrámem smrti, když se v něm právě nikdo nespaluje. Ale v koupelně, všiml jsem si, máme rozbitý ventilátor, musím to dát zítra spravit. Já jsem tam zatím dal provaz se smyčkou, aby šel ventilátor otevřít ze židle. Ta záclona támhle v rohu..." ukázal k oknu, "na kterou upozornil na Štědrý den Willi, už od té doby zase drží. Slyšíš ten krásný zpěv," ukázal na rádio, z něhož se nesla velká árie Lucie, "jaká je to pravda, že umírá chud, kdo nepoznal krásu hudby, kdepak je Rosana..."
Po večeři pan Kopfrkingl nebeskou políbil a řekl: "Pojď, nevýslovná, dřív než se svlékneme, připravíme koupelnu." A vzal židli a šli, dívala se na ně kočka. "Je tu horko," řekl pan Kopfrkingl v koupelně a postavil židli pod ventilátor, "asi jsem to přehnal s topením. Otevři ten ventilátor, drahá."
Když Lakmé vylezla na židli, pan Kopfrkingl jí pohladil lýtko, hodil jí smyčku na krk a s něžným úsměvem jí řekl: "Co abych tě, drahá, oběsil?" Usmála se na něho dolů, snad mu dobře nerozuměla, on se usmál též, kopl do židle a bylo to. V předsíni si vzal kabát, šel na německou kriminálku a do protokolu nadiktoval: "Udělala to zřejmě ze zoufalství. Měla židovskou krev a nesnesla žít po mém boku. Snad tušila, že se s ní dám rozvést, že se to nesrovnává s mou německou ctí." A sobě v duchu řekl: "Litoval jsem tě, drahá, litoval. Byla jsi skleslá, zamlklá, ovšem, jak by ne, ale já jsem tu oběť jako Němec přinést musil. Zachránil jsem tě, drahá, před utrpením, které by tě jinak čekalo. Jak bys byla, nebeská, s tou svou krví v tom novém šťastném, spravedlivém světě trpěla..."
Lakmé byla zpopelněna v Chrudimi u Slatiňan... a pan Kopfrkingl byl po svatodušních svátcích jmenován ředitelem pražského krematorie. Penzionoval pana Vránu ve vrátnici nádvoří, který tam seděl, že měl něco s játry, je už starý, myslil si, je tu už od mého nástupu bezmála před dvaceti léty, až si odpočine, paní Podzimkové, uklízečce, dal výpověď, vždyť se tu skoro bála, řekl si, ať ji zbavím strachu, té kletby... ale pana Dvořáka si ponechal, "víte, pane Dvořák, mně se na vás líbí, že nekouříte a nepijete..." řekl mu, "že jste abstinent..." a ponechal si také pana Pelikána a také pana Fenka si ještě nechal. Měl bych ho zachránit, myslil si někdy v ředitelně, sotva se drží na nohou. Když šel kolem vrátnice, pan Fenek plakal a zalézal jako pes.
    \section*{Analýza uměleckého textu}
    \begin{itemize}
        \item\textbf{zasazení výňatku do kontextu díla}
        \begin{itemize}
            \item 2. polovina knihy, skoro konec, kdy se pan Kopfrkingl snaží vymazat všechny vady na jeho životě proti nacistické ideologii - napůl židovská žena, stává se ředitelem krematoria, poté se pokusí zabít své dvě děti
        \end{itemize}
        \item\textbf{téma a motiv}
        \begin{itemize}
            \item \textbf{téma: }rozpad osobnosti, negativní vliv fašistické ideologie na člověka
            \item \textbf{motivy: }pohřební hudba, smrt (krematorium), židovská krev, německá čest (hlavně v ukázce), abstinence, kniha o Tibetu, dívka v černých šatech
        \end{itemize}
        \item\textbf{časoprostor}
        \begin{itemize}
            \item \textbf{čas: }těsně před WWII a nacistickou okupací - pozorujeme vliv nacismu na člověka (psáno člověkem, který žil v této době a byl homosexuál -> reakce na tehdejší situaci)
            \item \textbf{prostor: }Praha, reálný svět; situace v Praze při okupaci - židovská menšina, kluby pro čisté Němce (asi nejvíce v Praze)
        \end{itemize}
        \item\textbf{kompoziční výstavba}
        \begin{itemize}
            \item členěno do 15 kapitol
            \item přehledně chronologické, tradiční vyprávění, zápletka postupně graduje
        \end{itemize}
        \item\textbf{literární žánr a druh}
        \begin{itemize}
            \item\textbf{druh: }epika, próza
            \item\textbf{žánr: }psychologický román s horovými prvky
        \end{itemize}
        \item\textbf{vypravěč}
        \begin{itemize}
            \item nadosobní vševědoucí vypravěč, nevstupuje do děje svými komentáři a nehodnotí
            \item er-forma
        \end{itemize}
        \item\textbf{postava}
        \begin{itemize}
            \item pan Kopfrkingl - hlavní, fiktivní, přirozená, vývojová postava - zpočátku miluje svoji rodinu, podporuje kremaci, vše nazývá vznešenými názvy (čarokrásná, nevýslovná, \dots), vše spojuje s knihou o Tibetu, neřeší národnost, má spoustu židovských přátel, vše pořád opakuje (nepije, nekouří, je abstinent, \dots) - je jenom podivín; postupně vlivem svého německého přítele začíná propadat německé ideologii o čistém původu a když přijde okupace, okamžitě se přizpůsobí - spojí si nacistickou ideologii s tibetskou - rozpad osobnosti; nakonec se zblázní a myslí si, že se má stát dalším dalajlámou; hypotetická postava - nevíme, zda je jenom hloupý, nebo jestli touží po moci a využije pro to nacismus.
            \item postava slabocha - žije ve svém vnitřním světě a je ovlivněn násilnou ideologií - nacismem
        \end{itemize}
        \item\textbf{vyprávěcí způsoby}
        \begin{itemize}
            \item er-forma, chronologické vyprávění
        \end{itemize}
        \item\textbf{typy promluv}
        \begin{itemize}
            \item přímá řeč - uvozovky, nepřímá řeč (v ukázce)
        \end{itemize}
        \item\textbf{jazykové prostředky a jejich funkce ve výňatku}
        \begin{itemize}
            \item vypravěč spisovný jazyk
            \item pan Kopfrkingl v přímé řeči často až přehnaně spisovný, knižní výrazy; nadnesená pojmenování - čarokrásná, nebeská -- způsobuje to zvláštní nepříjemný pocit z postavy
            \item postupně se v knize objevuje více a více germanismů nebo přímo němčiny - jak se postava vyvíjí
            \item vypravěč kratší souvětí, pan Kopfrkingl občas dost rozsáhlá, obecně mluví mnohem více než ostatní postavy, opakuje ty samé věci pořád dokola
        \end{itemize}
        \item\textbf{tropy a figury a jejich funkce ve výňatku}
        \begin{itemize}
            \item figury zde moc nejsou - je to próza
            \item tropy
            \begin{itemize}
                \item silně metaforický jazyk, hlavně ve spojení se smrtí a krematoriem - kuchyně smrti, jízdní řád smrti, Chrám smrti (dalo by se považovat i za eufemismus)
                \item přirovnání - zalézal jako pes
                \item symboly - dívka v černých šatech (smrt, blížící se šílenství/tragedie)
            \end{itemize}
        \end{itemize}
    \end{itemize}
    \section*{Literárně historický kontext a autor}
    Ladislav Fuks se narodil po WWI, nebyl žid, ale viděl své židovské spolužáky, jak jsou odváděni do koncentračních táborů - strach.
    Byl homosexuál a bál se odhalení (bylo to bráno jako nemoc), celý život to tajil. Díky tomuto se dokázal vcítit do pocitů židů před WWII a popsat jejich pocity ve svých dílech.
    Vystudoval několik VŠ najednou (mezi nimi i psychologii) - pomohlo mu to v jeho dílech. První román až v 60. letech 20. století, kdy se autoři vraceli k dříve tabuizovaným tématům jako židovství a Holokaust.
    Spalovač mrtvol byl slavně zfilmován Jurajem Herzem.
    Jeho další díla jsou například Pan Theodor Mundstock (pražský žid trpící strachem z transportu, připravuje se na něj - jako jediný není nervózní, přejede ho náklaďák), Příběh kriminálního rady - netýká se židovství, ale má podobné hororové prvky.
    Další autor zabývající se židovskou tématikou je Arnošt Lustig (Modlitba pro Kateřinu Horovitzovou).
    Ve světě ve stejnou dobu antiutopie, existencialismus a beat generation.
\end{document}