\documentclass[11pt]{article}
\usepackage[a4paper, total={18cm, 27cm}]{geometry}
\usepackage[czech]{babel}
\usepackage[utf8]{inputenc}
\usepackage{pifont}
\usepackage{graphicx}
\usepackage{amssymb}

\begin{document}
    \begin{center}
        \underline{\textbf{\Huge Viktor Dyk - Krysař}}
    \end{center}
    \section*{Úryvek}
    Stála ve dveřích, jako tenkráte, když krysař lehké mysli a lehkého srdce vstupoval do města Hameln. Ale nesmála se. Horečně, upjatě hleděly její oči do nočního temna. Stála čekajíc, jak se zdálo; ale nečekala štěstí, nýbrž osud. Stála přibita na dveře jako na svůj kříž.

    Tiché kroky krysařovy ozvaly se a blížily. Horečka roztoužení nezachvěla však mladým tělem Agnes jako tolikrát; bezvládněji jen a rezignovaněji poddávala se neznámé tíze. A její oči nevzplály očekáváním: zíraly dále sklesle, bez naděje a víry, že by mohla kdy ustoupit noc.

    Ale toho všeho neviděl krysař. Jen bílý stín u dveří zahlédl. Zrychlil krok. Celá jeho touha rozletěla se k. Agnes. Na všechno pozapomenul v své touze: na zradu konšelů Hameln a jejich nesplněné sliby; na bledého kouzelníka, který sám sebe okouzlit nedovede, na toho, jenž smál se na výšinách chrámu svaté Trojice. Daleko je doktor Faustus. I ďábel je daleko té chvíle. Čeho by mohl hledati v tichém a zapadlém domě, který skrýval lásku krysařovu? Krysař šel. Šel, zavíraje oči. Klidně mohl tak kráčeti; znal příliš dobře cestu. V pravý okamžik rozevřel náruč, aby pojal do svých pevných ramenem rozkvetlé a teplé její tělo. Předevčírem, včera ještě zulíbaly jeho přivřená víčka žhavé rty a nahá ramena stiskla jeho hrdlo.

    V pravý okamžik rozevřel náruč. Ale stalo se něco podivného. Tělo, jehož se dotkl, nekladlo odporu; ale také života v něm nebylo. Nebo aspoň jakýs takýs chladný, znepokojivě mizející život. Nepocítil horkých polibků, nezaslechl šepot rozkoše a lásky.

    „Agnes!“ vydralo se z krysařových úst. Ale jeho slovo nedovedlo zapuditi strašidel.

    „Krysaři —,“ zašeptla ta, kterou měl v náručí. Ale hlas její zněl z tak nepřístupné dálky, že sotva krysař doslechl. Bylo to slovo, ne však odpověď.

    Na čase bylo, aby krysař rozevřel oči, jako rozevřel náruč. Nebylo pochybností; v svitu měsíčním choulila se zde u dveří žena, kterou miloval. Ale tvář její byla bledá, skoro zsinalá. Krysař rád by byl věřil, že tím vinen pouze odlesk měsíce. Ale zdálo se mu náhle, že slyší smích.

    Ďábel, který pokoušel krysaře, sestoupil pomalu s věže svaté Trojice. Přiblížil se lstivými kroky za krysařovými zády v zapadlý tento kout. Ze zahrady, za rozkvetlým bezem zněl jeho přitlumený chechtot. Jinak bylo ticho. Ale krysař rozuměl smíchu a rozuměl tichu. Věděl, že nutno se ptáti; věděl však též, že dostane se mu kruté odpovědi.

    Krysař nebyl však zbabělcem; odhodlal se přece k nebezpečné otázce. „Co je ti, Agnes?“

    Hlas krysařův zněl tiše a konejšivě, jak jen mohl krysařův hlas. A jeho ruka měkce se dotkla bezvládného a smutného těla milenčina. Oči Agnes rozevřely se široce. Bylo v nich něco neskonale plachého a bezradného. Neunikala, poněvadž nebylo možno uniknout. Nebránila se, poněvadž nebylo možno se ubrániti. Krysaře ranily tyto beznadějné oči, ustupující osudu. Sevřel Agnes prudčeji... tak prudce jako při prvním setkání.

    Ale cos bylo jiné než tenkráte: bolest ještě byla, ale rozkoš vyprchala.

    „Mluv přece, Agnes,“ naléhal krysař. „Pověz, co tě bolí. Tvé mlčení je krutější všeho, co bys mohla říci!“

    Ale krysař zarazil se při svých posledních slovech. Sevřelo to jeho srdce mučivou otázkou: Bylo tomu opravdu tak?

    Agnes učinila těžký a slabý pohyb někoho procitajícího z mdloby.

    „Pojď!“ zašeptala tiše a sklíčeně. Bylo zjevno, že složila ruce a dala se vésti fatalismem slabé ženy, prudkým, násilnickým proudem. Ale kam ji zanese?

    Šli. Také tuto cestu mohl krysař znáti, třeba kráčel temnotou. A přece klopýtnul; klopýtnuv, vzpomněl náhle na dlouhého Kristiána. Kupodivu! Dlouhého Kristiána jako by jindy nebylo. Krysař nepocítil nikdy žárlivosti; Kristián připadal mu něčím neškodným a marným, čím se zbytečno zabývat. Ale té chvíle ranila ho ta myšlenka. Jeho hrdost utrpěla krutou ránu. Nachýliv se k Agnes, kráčející před ním tmou, takže cítila jeho horký dech, otázal se prudce:

    „Je to Kristián?“

    Neodpověděla. Tápali oba němě tmou; a třeba ji znali z minulých nocí, stala se jim náhle cizinou.

    Hodiny bily na věži. Jednotvárně tento zvuk rozesnil krysaře na tři čtyři vteřiny. Ale pak ho probudil zoufalý a bezútěšný výkřik Agnes, která kdysi se tak usmívala do jarní noci:

    „Můj bože!“ 
    \section*{Analýza uměleckého textu}
    \begin{itemize}
        \item\textbf{zasazení výňatku do kontextu díla}
        \begin{itemize}
            \item ke konci díla
            \item Krysař se dozví, že je Agnes těhotná s Kristiánem 
            \item těsně před tím, než Agnes odejde na horu Koppel a zabije se
        \end{itemize}
        \item\textbf{téma a motiv}
        \begin{itemize}
            \item\textbf{téma: }láska, pomsta
            \item\textbf{motivy: }špatné lidské vlastnosti (Kristián, konšelé)
            \begin{itemize}
                \item na motivy staré saské pověsti o krysaři
            \end{itemize}
        \end{itemize}
        \item\textbf{časoprostor}
        \begin{itemize}
            \item \textbf{čas: }minulost - středověk (blíže neurčená)
            \item \textbf{prostor: }reálný, německé městečko Hammeln
        \end{itemize}
        \item\textbf{kompoziční výstavba}
        \begin{itemize}
            \item členěno do 26 kapitol - různě dlouhé (půl strany - několik stran)
            \item chronologická kompozice
        \end{itemize}
        \item\textbf{literární žánr a druh}
        \begin{itemize}
            \item \textbf{druhy: }epika, próza
            \item \textbf{žánr: }novela
        \end{itemize}
        \item\textbf{vypravěč / lyrický subjekt}
        \begin{itemize}
            \item nadosobní, vševědoucí vypravěč
            \item er-forma
            \item odhaluje čtenáři děj i pocity všech postav
        \end{itemize}
        \item\textbf{postava}
        \begin{itemize}
            \item\textbf{Krysař (hlavní postava)} - Záhadná postava (nevíme odkud je, ani jak je starý, lze považovat za hypotetickou postavu) s kouzelnou píšťalou, kterou dokáže ovládat (nejenom) krysy. Cestuje po městech a za peníze je zbavuje krys. Navenek působí tvrdě a necitlivě, ale zamiluje se do Agnes. Kvůli tomu, že za práci nedostal slíbenou odměnu a žalem po smrti Agnes odvede všechny obyvatele Hammeln do propasti.
            \item\textbf{Agnes (hlavní postava)} - Krásná dívka z Hammeln. Má přítele Kristiána, ale okamžitě se zamiluje do Krysaře. Otěhotní s Kristiánem a ze zoufalství odejde na vrch Koppel a skočí do propasti (snaží se dostat do ráje = Sedmihradská země)
            \item\textbf{Kristián (vedlejší postava)} - Mladík, který pracuje v dílně svého strýce. Přeje strýci smrt, aby zdědil jeho dílnu a majetek a stal se tím váženým občanem.
            \item\textbf{Sepp Jorgen (vedlejší postava)} - Místní rybář, naivní, ostatní si z něj dělají legraci. Je pomalejší než ostatní (všechno mu dojde až druhý den). Když Krysař zapíská na píšťalu a odvede všechny do propasti, Sepp jako jediný přežije. Zvuk píšťaly ho začne ovládat až později, ale z její moci ho vytrhne pláč dítěte, kterému zemřela matka.
        \end{itemize}
        \item\textbf{typy promluv}
        \begin{itemize}
            \item převážně vlastní přímá řeč - \textit{„Pověz, co tě bolí. Tvé mlčení je krutější všeho, co bys mohla říci!“}
        \end{itemize}
        \item\textbf{jazykové prostředky a jejich funkce ve výňatku}
        \begin{itemize}
            \item spisovný jazyk
            \item použití přechodníků - \textit{čekajíc, zavíraje oči, ...} \rotatebox[origin=c]{270}{$\Rsh$}
            \item archaismy (\textit{konšelé, krysař}) - je to příběh z minulosti \ding{219} větší autenticita
            \item nespisovný jazyk zřídka (\textit{jakýs takýs})
            \item krátké a jednoduché věty, krátká souvětí - posiluje to napětí, přehlednější, děj rychleji ubíhá
        \end{itemize}
        \item\textbf{tropy a figury a jejich funkce ve výňatku}
        \begin{itemize}
            \item\textbf{figury:} inverze (nepřirozený slovosled) a elipsa - \textit{Krysař rád by byl věřil, že tím vinen pouze odlesk měsíce.} - působí to knižněji
            \item\textbf{tropy:}
            \begin{itemize}
                \item text sám o sobě není moc bohatý na básnické obrazy
                \item přirovnání - \textit{Stála přibita na dveře jako na svůj kříž.}
                \item symbol - krysařova flétna (symbol moci, magična), dítě - symbol příchodu něčeho nového
                \item alegorie - Příběh alegoricky zobrazuje lidi se špatnými vlastnostmi jako krysy, které přišel vyhnat Krysař. Jeho trestu unikne pouze Sepp Jorgen, který je jako jediný čistý a bez negativních vlastností.
            \end{itemize}
        \end{itemize}
    \end{itemize}
    \section*{Literárně historický kontext a autor}
    Viktor Dyk byl český básník, prozaik, dramatik a politik.
    Reflexivní lyrik, romantik.
    Po vzniku Československa redaktor Národních listů.
    Krysař je spíše z konce jeho tvorby, zpočátku psal spíše dekadentní poezii (A porta inferi).
    Krysař je jeho jediná významná próza, psal hlavně poezii (Síla Života, Marnosti, Milá sedmi loupežníků, válečná tetralogie - vlastenecká poezie)
    Napsal také drama Zmoudření Dona Quijota.
    Díla jsou známá ztrátou iluzí, skepsí, někdy i ironií.

    Tvořil na přelomu 19. a 20. století.
    Krysař - novoklasicismus s prvky romantismu.
    Pod vlivem anarchismu, antimilitarismu a civilismu, řazen mezi generaci buřičů.
    Kritizovali soudobou konzervativní společnost, žili bohémským životem.
    Má blízko ke Gellnerovi, Neumannovi a jiným autorům z Moderní revue.
    Dílo dobře přijato, dodnes oblíbené.
    Několikrát zfilmováno, muzikál Krysař.
    \section*{Zdroje}
    \begin{verbatim}
        Literatura v kostce pro SŠ, Marie Sochrová
        https://cs.wikipedia.org/wiki/Krysař_(kniha)
        https://www.rozhlas.cz/ctenarskydenik/autori/_zprava/dyk-viktor--935210
        https://cs.wikipedia.org/wiki/Anarchističtí_buřiči
        http://praktik.caloris.cz/literatura/lit-krysar.html
    \end{verbatim}
    \end{document}