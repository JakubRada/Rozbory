\documentclass[11pt]{article}
\usepackage[a4paper, total={18cm, 27cm}]{geometry}
\usepackage[czech]{babel}
\usepackage[utf8]{inputenc}

\begin{document}
    \begin{center}
        \underline{\textbf{\Huge Terry Pratchett - Soudné sestry}}
    \end{center}
    \section*{Úryvek}
    Voda se maličko zčeřila, pak se uklidnila a nakonec se s tichým žbluňkáním a bubláním její hladina pozvedla a změnila v hlavu. Magráta upustila mýdlo.
Nebyla to ošklivá hlava, možná trochu krutá kolem očí a trochu zobanovitá kolem nosu, ale celkem hezká, takovým tím tvrdším způsobem. Na to mne bylo nic udivujícího. Protože démon tady nebyl, ale pouze promítal svůj obraz do reality, mohl si na něm dát záležet. Pomalu se otáčel dokola a podobal se černé soše, od jejíhož povrchu se odráží měsíční
světlo.
„No?“ řekl.
„Kdo jsi?“ zaútočila na něj Bábi přímo.
Hlava se otočila k ní.
„Mé jméno se nedá v tvém jazyce vyslovit, ženská,“ řekla.
„O to se nestarej, to posoudím sama,“ odbyla ho Bábi a dodala: „A neodvažuj se mi říkat ,ženská’.“
„Dobrá. Jmenuji se WxrtHltl-jwlpklz,“ oznámil jí démon s mazaným úšklebkem.
„A kde ses potuloval, když se rozdávaly samohlásky, co? Zapomněls je za dveřmi?“ neodpustila si Stařenka Oggová.
„Tak dobrá, pane -“ Bábi zaváhala jen na zlomek vteřiny „-WxrtHltl-jwlpklzi, předpokládám, že by vás zajímalo, proč jsme vás dnes večer vyvolaly.“
„Nečeká se od vás, že budete říkat něco takového,“ prohlásil démon. „Čeká se, že řeknete -“
„Sklapni. Varuji tě. Máme meč Magie a osmiúhelník Ochrany.“
„Když myslíte. Mně ovšem připadají jako stará valcha a hůl z nějakého tvrdého dřeva,“ démon se odporně zašklebil.
Bábi vrhla rychlý pohled stranou. V rohu prádelny byla navršena naštípaná polena a před hromadou stála těžká koza na řezání dřeva. Aniž spustila oči z démona, udeřila svou holí do silných trámků.
Hrobové ticho, které následovalo, přerušil jen tichý zvuk obou polovin hladce přeseknuté kozy na řezání dřeva, jež se pomalu oddělily a převrhly na hromadu drobného dřeva.
Démonova tvář se ani nepohnula.
„Můžete mi položit tři otázky,“ prohlásila hlava.
„Děje se v království něco zvláštního?“ zeptala se Bábi.
Zdálo se, že o tom hlava přemýšlí.
„A žádné lhaní,“ upozorňovala Magráta, „nebo tě čeká tady ten rýžový kartáč!“
„Myslíš zvláštnějšího než obvykle?“
„Tak ven s tím,“ povzbuzovala ho Stařenka. „Začínají mě tady zábst nohy!“
„Ne, nic takového se neděje.“
„Ale my jsme to přece cítily -“ začala Magráta.
„Počkej, zadrž,“ řekla Bábi. Rty se jí neslyšně pohybovaly. Démoni se často chovali jako géniové nebo profesoři filozofie - když jste své otázky nevyjádřili naprosto přesně a jednoznačně, nacházeli neobyčejné potěšení v tom, že vám poskytli přesnou, vyčerpávající, ale co možná nejvíce zavádějící odpověď.
„Objevilo se v království něco důležitého, co v něm předtím nebylo?“ odvážila se nakonec.
„Ne.“
Tradice říkala, že démonovi se smějí položit jen tři otázky. Bábi se nejdřív v duchu pokusila zformulovat tu třetí tak, aby musel odpovědět jednoznačně. Pak ale dospěla k názoru, že není důvod, proč by se měla pouštět do hazardu.
„Tak už k sakru vyklop, co se děje, ano?“ prohlásila potom nahlas. „A nepokoušej se z toho zase vykroutit, nebo tě prostě uvařím.“
Zdálo se, že démon zaváhal. Takhle to ještě neznal.
„Magráto, hoď sem hrst toho dřeva, buď tak hodná,“ nadhodila Bábi přes rameno.
    \section*{Analýza uměleckého textu}
    \begin{itemize}
        \item\textbf{zasazení výňatku do kontextu díla}
        \begin{itemize}
            \item první polovina díla, čarodějnice zjistily, že je v Lancre něco špatně a snaží se zjistit co
        \end{itemize}
        \item\textbf{téma a motiv}
        \begin{itemize}
            \item\textbf{téma: }parodie na typické postupy fantasy žánru, narážka na Hamleta a Macbetha od Shakespeara, částečně satirizuje i reálný svět
            \item\textbf{motivy: }čarodějnice, kouzla, šašek, duch, démon, divadlo, král
        \end{itemize}
        \item\textbf{časoprostor}
        \begin{itemize}
            \item\textbf{čas: }nespecifikovaný čas, v pozdějším ději je čas kouzlem posunut o 15 let dopředu
            \item\textbf{prostor: }fantastický svět - Zeměplocha (celá knižní série z tohoto světa), svět plný magie a nadpřirozených stvoření
        \end{itemize}
        \item\textbf{kompoziční výstavba}
        \begin{itemize}
            \item text není členěn na kapitoly, jednotlivé přiběhové linie jsou oddělené vynechaným řádkem
            \item spoustu poznámek pod čarou, autor vysvětluje okolnosti z jiných knih, nebo své neologismy
            \item chronologická, částečně paralelní - probíhá několik dějových linek, které probíhají ve stejný čas nebo na sebe navazují
        \end{itemize}
        \item\textbf{literární žánr a druh}
        \begin{itemize}
            \item\textbf{druh: }próza, epika
            \item\textbf{žánr: }román, fantasy
        \end{itemize}
        \item\textbf{vypravěč / lyrický subjekt}
        \begin{itemize}
            \item nadosobní vševědoucí vypravěč - zprostředkovává všechny informace o ději, nehodnotí
            \item er-forma
        \end{itemize}
        \item\textbf{postava}
        \begin{itemize}
            \item Magráta Česneková, Stařenka Oggová, Bábi Zlopočasná - hlavní, fiktivní, nadpřirozené, kladné - Magráta je nejmladší, věří v tradice a nefunkční okultní předměty a postupy, zamiluje se do královského šaška; Stařenka Oggová je nejstarší, má obrovskou rodinu, skoro až klan, holduje alkoholu a radovánkám; Bábi Zlopočasná - prostředně stará, rázná, odvážná, z trojice budí nejvíce strach; zjistí, že je něco špatně (příroda se bouří) v království a snaží se to napravit - pomocí skoro nemožného kouzla posunou čas o 15 let dopředu, aby mladý král v exilu mohl nahradit stávajícího tyrana
            \item šašek - vedlejší, fiktivní, přirozená - depresivní, velmi chytrý, může říkat jenom předem schválené vtipy, protože humor je vážná věc, zamiluje se do Magráty, nakonec se stává králem poté co Tomjan odmítne svůj nárok
            \item Tomjan - hlavní, fiktivní, přirozená - syn zabitého krále, čarodějky ho předaly kočovným hercům, chlapec si zamiluje divadlo a stane se nejlepším hercem, už se nechce stát králem, pomocí jeho divadelní skupiny odhalí královu vraždu (jako v Hamletovi)
            \item Lord Felmet - vedlejší, fiktivní, přirozená - nový král Lancre, snadno manipulovatelný, trochu šílený, byl svou krutou ženou donucen zabít předchozího krále a stal se novým králem, i když vlastně vládne jeho žena (jako Macbeth), trápí ho silné výčitky (Hamlet) a krvácí mu pořád ruce
            \item pak je tam trpaslík, duch, démon, \dots
        \end{itemize}
        \item\textbf{vyprávěcí způsoby}
        \begin{itemize}
            \item er-forma
            \item situační a jazyková komika
        \end{itemize}
        \item\textbf{typy promluv}
        \begin{itemize}
            \item přímá řeč - „Dobrá. Jmenuji se WxrtHltl-jwlpklz,“
        \end{itemize}
        \item\textbf{jazykové prostředky a jejich funkce ve výňatku}
        \begin{itemize}
            \item spisovný jazyk ve vypravěčových pasážích
            \item v promluvách občas nespisovné slovo - zapomněls
            \item neologismy - lemtat (pít), ... - vždy vysvětlené v poznámkách pod čarou
            \item expresivní slova - vyklop to
            \item jazyková komika - jména (Tomjan - nemohli se rozhodnot, jestli to bude Tomáš nebo Jan; Zlopočasná; Vínozpěv, Mášrecht)
            \item v promluvách krátké věty, vypravěč občas dlouhá souvětí - akčnější děj, popisy mohou být detailnější
        \end{itemize}
        \item\textbf{tropy a figury a jejich funkce ve výňatku}
        \begin{itemize}
            \item přirovnání - démoni se často chovali jako géniové nebo profesoři filozofie
        \end{itemize}
    \end{itemize}
    \section*{Literárně historický kontext a autor}
    Terry Pratchett je anglickým autorem celé fantasy knižní série Úžasná Zeměplocha, ve které paroduje Tolkiena, Howarda nebo Lovecrafta.
    V 70. letech nebyl moc úspěšný, živil se jako novinář. V 80. letech obrovský úspěch a ohlasy na Zeměplochu, od té doby celebrita.
    Nechtěl dávat moc informací o svém životě, protože to čtenáře, podle jeho slov, nezajímá.
    Byl jmenován rytířem. Zemřel na Alzheimera, všechna nedokončená díla byla na jeho přání zničena.
    Soudné sestry jsou šestou knihou ze série, která čítá 41 knih. Jiná známá díla ze série jsou např. Barva kouzel, Čaroprávnost.
    Nejvíce tvořil od 80. let do současnosti.
    Dílo skvěle přijato, v 90. letech nejprodávanější britský spisovatel, díla několikrát zdivadelněna i zfilmována.
    \end{document}