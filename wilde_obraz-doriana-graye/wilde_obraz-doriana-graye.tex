\documentclass[11pt]{article}
\usepackage[a4paper, total={18cm, 27cm}]{geometry}
\usepackage[czech]{babel}
\usepackage[utf8]{inputenc}

\begin{document}
    \begin{center}
        \underline{\textbf{\Huge Oscar Wilde - Obraz Doriana Graye}}
    \end{center}
    \section*{Úryvek}
    »Vidět mou duši!« zašeptal Dorian Gray; vyskočil z pohovky a téměř zbělel hrůzou.
»Ano,« odpověděl vážně Hallward a v hlase mu zněl hluboký tón zármutku, »vidět vaši
duši. Ale to může jen Bůh.«
Mladšímu muži se vydral ze rtů trpký výsměšný smích. »I vy ji dneska uvidíte!« zvolal a
zvedl ze stolu lampu. »Pojďte! Je dílem vašich rukou, tak proč byste se na ni nemohl podívat?
A pak o tom všem můžete vykládat, když si budete přát. Ale nikdo vám neuvěří. A i kdyby
vám věřili, tím víc bych se jim pro to líbil. Znám dnešní dobu líp než vy, přesto, že o ní pořád
tak do omrzení žvaníte. Pojďte, povídám. Už dost jste se o mravní zkáze naslýchal. Teď se jí
podíváte rovnou do tváře.«
V každém slovu, jež vyřkl, zněla šílená pýcha. Dupal na podlahu a zase už se choval jako
neukázněný chlapec. Pociťoval strašnou radost při pomyšlení, že s ním má někdo jiný sdílet
jeho tajemství a že muž, který namaloval portrét, v němž je prapůvod celé jeho hanby, má v
sobě po celý zbytek života nést jako těžké břímě odpornou vzpomínku na to, co udělal.
»Ano,« pokračoval, přistupuje k Hallwardovi a hledě mu upřeně do přísných očí, »ukážu
vám svou duši. Uvidíte to, co podle vašich představ může vidět jen Bůh.«
Hallward ucouvl. »Rouháte se, Doriane!« zvolal. »Takové věci nesmíte říkat. Jsou hrozné a
nemají smysl.«
»Myslíte?« Dorian se opět zasmál.
»Vím to. Pokud jde o to, co jsem vám dnes říkal, při tom jsem myslel jen na vaše dobro.
Víte, že jsem odjakživa váš oddaný přítel.«
»Nedotýkejte se mne! Dopovězte, co máte na srdci.«
Malířovou tváří se mihl křečovitý záchvěv bolesti. Na okamžik se odmlčel a zaplavil ho
prudký pocit lítosti. Jaké má konec konců právo, aby se Dorianu Grayovi vměšoval do života?
Udělal-li Dorian jen desetinku toho, co se o něm proslýchá, jak mnoho za to jistě vytrpěl!
Hallward se napřímil, přešel ke krbu a stanul u něho, hledě na hořící polena, na popel z nich,
podobný jinovatce, a na jejich dřeň tepající plamenem.
»Čekám, Basile,« řekl mladý muž tvrdým, jasným hlasem.
Hallward se otočil. »Tohle mám na srdci!« zvolal. »Musíte mi nějak zodpovědět ta hrozná
obvinění, která proti vám lidé vznesli. Když mi řeknete, že jsou od začátku do konce naprosto
nepravdivá, uvěřím vám. Popřete je, Doriane, popřete je! Copak nevidíte, co prožívám? Bože
můj! Neříkejte mi, že jste člověk špatný a zkažený a nestoudný.«
Dorian Gray se usmál. Rty se mu pohrdavě zvlnily. »Pojďte nahoru, Basile,« řekl mírně.
»Vedu si deník o svém životě, den po dni, a ten zůstává pořád v místnosti, v které je psán.
Ukážu vám ho, když půjdete se mnou.«
»Půjdu s vámi, Doriane, když si to přejete. Vidím, že jsem viz stejně zmeškal vlak. To
nevadí. Mohu jet zítra. Ale nechtějte po mně, abych teď v noci něco četl. Nechci nic jiného
než přímou odpověď na svou otázku.«
»Tu dostanete nahoře. Tady vám ji dát nemohu. A nebudete muset číst dlouho.«
    \section*{Analýza uměleckého textu}
    \begin{itemize}
        \item\textbf{zasazení výňatku do kontextu díla}
        \begin{itemize}
            \item druhá polovina knihy, po několika letech hříchů, kdy už je obraz značně znetvořený a Basil vyjádřil přání dozvědět se pravdu do Dorianových činech; těsně před tím, než Dorian ukáže Basilovi obraz a zabije ho
        \end{itemize}
        \item\textbf{téma a motiv}
        \begin{itemize}
            \item\textbf{téma: }manipulace jiným člověkem, dvojí život (Dorian, navenek působí mladě a vznešeně, ve skutečnosti je zkažený a páchá zločiny), kontrast krásy a zkaženosti, narcisismus, (hodnoty tehdejší společnosti)
            \item\textbf{motivy: }duše, hřích, krása, mládí, smrt, magično
        \end{itemize}
        \item\textbf{časoprostor}
        \begin{itemize}
            \item\textbf{prostor: }reálný, konkrétní Londýn - vybraná vysoká společnost (kluby, opera, divadlo, večírky, lovecký zámeček), ale i místa spojená se spodinou (opiová doupata, temné uličky) - navazuje na dvojitou osobnost Doriana
            \item\textbf{čas: }Viktoriánská éra - 2. polovina 19. století - autorova současnost
            \begin{itemize}
                \item v druhé polovině je zhuštění, kdy se v jedné kapitole shrne několik let Dorianových zločinů a změn zájmů a sbírek
                \item poté děj pokračuje už se znetvořeným obrazem
                \item vyjadřuje svůj cynický postoj k tehdejší společnosti
            \end{itemize}
        \end{itemize}
        \item\textbf{kompoziční výstavba}
        \begin{itemize}
            \item uspořádáno do kapitol (při prvním vydání byla zkrácena na 13 kapitol kvůli na tu dobu nemravným scénám, ale později vyšla původní necenzurovaná verze, která má 20 kapitol)
            \item na začátku delší verze je známá předmluva - po kritice kvůli nemravnosti zde oslovuje kritiky a vysvětluje, že umění není morální nebo nemorální, ale jde o krásu a estetično
            \item chronologické vyprávění, v druhé polovině je kratší časový skok, kde je několik let popsáno v jedné kapitole
        \end{itemize}
        \item\textbf{literární žánr a druh}
        \begin{itemize}
            \item\textbf{druh: }epika, próza
            \item\textbf{žánr: }román (filosofický, alegorický)
        \end{itemize}
        \item\textbf{vypravěč / lyrický subjekt}
        \begin{itemize}
            \item nadosobní vševědoucí vypravěč - odtajňuje nám pocity postav, nehodnotí
            \item er-forma
        \end{itemize}
        \item\textbf{postava}
        \begin{itemize}
            \item Dorian Gray - hlavní, fiktivní, přirozená, vývojová - zpočátku je mladý, nezkažený, krásný, netrpí žádnými špatnými vlastnostmi, kvůli Henrymu se stane posedlý svou krásou, pro svůj prospěch udělá cokoli; postupně se pod Henryho vlivem kazí a páchá více a více hříchů, zabije se kvůli němu Sybil Vane a jemu to po promluvě s Henrym nepřijde špatné, až nakonec zavraždí Basila a vydírá svého bývalého přítele, aby zahladil stopy; na konci už si uvědomuje, jak je zkažený a chce se změnit, ale ve vzteku, že mu to nejde probodne obraz a umírá; zpočátku je neutrální postavou, ale mění se v zápornou (Henryho vlivem)
            \item Henry Wotton - hlavní, fiktivní, přirozená, statická, záporná (ztělesnění nemorálnosti) - lord plný filozofických myšlenek, provádí experimenty na Dorianovi, kterému to nedochází, a přesvědčuje ho k životu zasvěcenému požitkům; záměrně Dorianovi říká, že záleží jen na kráse, kterou ovšem v budoucnu ztratí a způsobuje tím Dorianovu proměnu ve špatného člověka; celou dobu si to uvědomuje, ale Dorian je ideální předmět k pokusům
            \item Basil Hallward - hlavní, fiktivní, přirozená, statická, neutrální - úspěšný geniální malíř, od začátku se snaží zabránit Henrymu v kazení Doriana, ale není moc schopný, kritizuje Dorianovo špatné chování, po Dorianově obrazu už není schopen udělat žádné další dílo, je do něj zamilován, ale nemůže to říct
        \end{itemize}
        \item\textbf{vyprávěcí způsoby}
        \begin{itemize}
            \item er-forma
        \end{itemize}
        \item\textbf{typy promluv}
        \begin{itemize}
            \item přímá řeč (vlastní i nevlastní - Jaké má konec konců právo, aby se Dorianu Grayovi vměšoval do života? Udělal-li Dorian jen desetinku toho, co se o něm proslýchá, jak mnoho za to jistě vytrpěl!)
        \end{itemize}
        \item\textbf{jazykové prostředky a jejich funkce ve výňatku}
        \begin{itemize}
            \item spisovný jazyk
            \item knižní slova - břímě, nestoudný, vyřkl, stanul - dříve tato slova knižní nebyla (používala se běžně)
            \item historismy - drožka (není v ukázce)
            \item přechodníky - hledě
            \item občas francouzské výrazy a názvy - součást kultury vysoké společnosti
            \item v promluvách postav jednoduché věty, nebo krátká souvětí (lord Henry občas i dlouhá souvětí při filozofických úvahách); vypravěč rozsáhlá souvětí, hlavně v pasážích bez interakcí mezi postavami (dlouhé popisy, \dots)
        \end{itemize}
        \item\textbf{tropy a figury a jejich funkce ve výňatku}
        \begin{itemize}
            \item metafory - popel podobný jinovatce, tepající plamen
            \item symboly - žlutá kniha (symbol Henryho vlivu, manipulace, zkaženosti), 
        \end{itemize}
    \end{itemize}
    \section*{Literárně historický kontext a autor}
    Spisovatel irského původu, zabýval se estetikou a literaturou. Byl homosexuál a byl za to vězněn (odráží se v knize ve vztahu Basila s Dorianem).
    Představitel dekadence a esteticismu, měl provokativní styl života - ODG to odráží.
    ODG je jeho jediný román. Autocenzuroval ho, protože se bál, že bude moc skandální (když znovu připravoval prodlouženou verzi).
    Psal i básně (Básně - sbírka dekadentních básní), dramata (Jak je důležité míti Filipa, Salome), pohádky (Šťastný princ a jiné pohádky).
    Tvořil ke konci 19. století, jeho současníky jsou např. prokletí básníci (Verlaine, Rimbaud) u nás Karel Hlaváček (taky dekadent).
    Dobově špatně přijato (kritiky) - bylo bráno jako velmi nemorální - proto vydal další verzi s předmluvou, kde se snažil tuto kritiku zvrátit.
    V dnešní době považováno za jedno z nejdůležitějších děl té doby, které ovlivnilo mnoho děl poté.
    Mnohokrát ztvárněno filmově či divadelně.
\end{document}