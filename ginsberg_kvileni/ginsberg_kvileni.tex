\documentclass[11pt]{article}
\usepackage[a4paper, total={18cm, 27cm}]{geometry}
\usepackage[czech]{babel}
\usepackage[utf8]{inputenc}

\begin{document}
    \begin{center}
        \underline{\textbf{\Huge Allen Ginsberg - Kvílení}}
    \end{center}
    \section*{Úryvek}
    \begin{center}
Viděl jsem nejlepší hlavy své generace zničené šílenstvím, hystericky obnažené a o hladu,

vlekoucí se za svítání černošskými ulicemi a vztekle shánějící dávku drogy, hipstery s

andělskými hlavami, celé žhavé po prastarém nebeském kontaktu s hvězdným dynamem ve

strojovně noci,

kteří v bídě a v hadrech a se zapadlýma očima a podnapilí vysedávali a kouřili v nadpřirozené

temnotě bytů se studenou vodou, vznášeli se přitom nad vrcholky velkoměst a kontemplovali

o džezu,

kteří odhalili mozky Nebesům pod nadzemní dráhou a spatřili mohamedánské anděly, jak se

potácejí prozřelí po střechách činžáků,

kteří procházeli univerzitami se zářivýma studenýma očima, měli halucinace o Arkansasu a

tragédii Blakeova světla mezi válečnými vědátory,

kteří vyloučili z akademie pro bláznovství \& publikaci obscénní ódy na oknech lebky,

kteří křehli zimou v neholených pokojích ve spodním prádle, pálili peníze v koších na

odpadky a naslouchali přes zeď Hrůze,

které zkopali do přirození, když se vraceli přes Laredo s opaskem marihuany pro New York,

kteří polykali oheň v popatlaných hotelech nebo pili terpentýn v Rajské uličce, smrt, nebo noc

co noc očistcovali svá torza

pomocí snů, pomocí drog, pomocí bdělých nočních můr, alkoholu, ocasu a nekonečných

prcponků,

nesrovnatelné slepé ulice hrůzných mračen a blesků v duši, vyskakujících na telegrafní tyče v

Kanadě \& Patersonu a osvětlujících celý ten nehybný svět Času mezi nimi,

peyotlová ztuhlost hal, dvorek, zelený strom, hřbitovní úsvity, opilost vínem nad vrcholky

střech, výkladní skříně, okresy marihuanových čundrů, neónově blikající dopravní světla,

záchvěvy slunce a měsíce a stromů za halasných zimních soumraků v Brooklynu, popelnicové

tlachy a konejšivé královské světlo myšlenky,

kteří se připevnili k vagónům podzemní dráhy na nekonečných jízdách z Battery do svatého

Bronxu a křísili se benzedrinem, dokud je rámusení kol a dětí neshodilo se zkřivenými ústy,

mozky vyklepanými k prázdnotě a zbavenými oslnivosti do ponurého světla zoologické

zahrady,

kteří tonuli celou noc v podmořském světle u Bickforda, pak se vynořili a proseděli celé

odpoledne nad zvětralým pivem u opuštěného Fugazziho a naslouchali třeskotu osudu na

vodíkové hrací skříni,

kteří hovořili sedmdesát hodin bez přestání, od parku k silnici, k baru, k Bellevue, k muzeu, k

Brooklynskému mostu,

ztracený regiment platonických causeurů, vrhajících se z požárních žebříků, z okenních rámů,

z Empire State Building, z měsíce,

klábosících, ječících, zvracejících a šeptajících fakta a vzpomínky a anekdoty a přípitky na

kuráž a šoky z nemocnic a kriminálů a válek,

když předtím s jiskřivýma očima naprosto neodvolatelně vyzvraceli během sedmi dnů a nocí

všechen intelekt, maso pro Synagógu házené na dlažbu,

kteří se propadli do nicoty zenového New Jersey a zanechali za sebou viditelnou stopu

dvojsmyslných pohlednic radnice z Atlantic City,

trpěli asijským potem a tangerským drcením kostí a čínskými migrénami, když jim v

pochmurně zařízeném newarském sále zarazili další přísun drog,

kteří se toulali o půlnoci sem a tam po nákladovém nádraží, lámali si hlavu kam jít a odešli a

nezanechali žádná zlomená srdce,

kteří si zapalovali cigarety v dobytčácích, dobytčácích, v dobytčácích rachotících sněhem k

zapadlým farmám v pradědečkovské noci,

kteří studovali Plotina, Poea, sv. Jana od Kříže, telepatii a bopovou kabalu, poněvadž vesmír

se instinktivně zachvíval u jejich nohou v Kansasu,

kteří samotařili na ulicích v Idahu a hledali vizionářské indiánské anděly, kteří byli

vizionářskými indiánskými anděly,

kteří si mysleli, že jsou jen šílení, když se Baltimore třpytil v nadpřirozené extázi,

kteří odfrčeli v limuzínách s oklahomským Číňanem z popudu maloměstského deště v

pouličním světle zimní půlnoci,

kteří se potloukali Houstonem hladoví a osamělí a odpuštění a hledali džez nebo sex nebo

polívku a následovali zářivého Španěla, aby si pokonverzovali o Americe a Věčnosti, marná

námaha, a tak sedli na loď do Afriky,

kteří zmizeli v mexických vulkánech a nezanechali po sobě nic než stín montérek a lávu a

popel poezie roztroušený v chicagském krbu,

kteří se znova objevili na Západním pobřeží a vyšetřovali FBI, s plnovousy a v šortkách, s

velkýma pacifistickýma očima, tak smyslnýma v jejich snědé pleti, a rozdávali

nesrozumitelné letáky,

kteří si vypálili cigaretami díry do paží na protest proti narkotickému tabákovému dýmu

kapitalismu,

kteří kolportovali superkomunistické pamflety na Union Square, plakali a svlékali se, zatímco

sirény z Los Alamos je uječely a uječely i Wall Street a píšťaly od přívozu na Staten Island

ječely také,

kteří se s pláčem zhroutili v bílých tělocvičnách, nazí a rozechvělí před soukolím jiných

koster,

kteří pokousali detektivy na krku a s gustem kvičeli v policejních antonech, neboť nespáchali

žádný jiný zločin kromě své vlastní divoké odvazové pederastie a opojení,

kteří kvíleli na kolenou v podzemce a které odvlekli ze střechy mávající genitáliemi a

rukopisy,

kteří se nechali mrdat do prdele od svatouškovských motocyklistů a vřeštěli radostí,

kteří kouřili ocasy a nechali si je kouřit od těch lidských serafínů, námořníků, něžnosti

atlantické a karibské lásky,

kteří si ráno i večer v růžových zahradách a na trávnících veřejných parků a hřbitovů a štědře

utrušovali semeno každému, kdo přišel,

kteří donekonečna škytali a pokoušeli se zachechtat, ale skončili uvzlykaní za přepážkou v

Turecké lázni, když je plavovlasý \& nahý anděl přišel proklát mečem,
    \end{center}
    \section*{Analýza uměleckého textu}
    \begin{itemize}
        \item\textbf{zasazení výňatku do kontextu díla}
        \begin{itemize}
            \item začátek první části básně
        \end{itemize}
        \item\textbf{téma a motiv}
        \begin{itemize}
            \item\textbf{téma: }kritika tehdejší americké společnosti, materialismus, zármutek a stěžování si, že nejlepší lidé generace jsou zničeni společností a musí hledat útěchu v drogách a alkoholu
            \item\textbf{motiv: }alkohol, drogy, homosexualita, šílenství, sny
        \end{itemize}
        \item\textbf{časoprostor}
        \begin{itemize}
            \item 50. léta v USA (San Francisco) - reakce na stav americké společnosti v té době
        \end{itemize}
        \item\textbf{kompoziční výstavba}
        \begin{itemize}
            \item báseň se skládá ze 3 částí
            \item na začátku je věnování - Jacku Kerouacovi, Williamu Burroughsovi a Nealu Cassadymu; celé je to dedikováno Carlu Solomonovi (kamarád z blázince)
            \item na konci je poznámka pod čarou - svatost každodenních věcí - téma všednosti, souvisí s dalšími básněmi sbírky
        \end{itemize}
        \item\textbf{literární žánr a druh}
        \begin{itemize}
            \item\textbf{druh: }poezie, lyricko-epické
            \item\textbf{žánr: }poema
        \end{itemize}
        \item\textbf{vypravěč / lyrický subjekt}
        \begin{itemize}
            \item lyrický subjekt je sám autor
            \item zprostředkovává svědectví o generaci, bezprostřední, vyjadřuje se otevřeně
            \item ich-forma
        \end{itemize}
        \item\textbf{postava}
        \begin{itemize}
            \item Carl Solomon - reálná postava - beatnický přítel Ginsberga, potkali se v blázinci
            \item není explicitně uveden v básni, ale je mu to věnováno
        \end{itemize}
        \item\textbf{vyprávěcí způsoby}
        \begin{itemize}
            \item ich-forma
            \item proud vědomí - byl pod vlivem 3 drog, halucinogenní pocity a představy
        \end{itemize}
        \item\textbf{typy promluv}
        \begin{itemize}
            \item je to poezie
        \end{itemize}
        \item\textbf{veršová výstavba}
        \begin{itemize}
            \item volný verš
            \item nerýmuje se to
            \item veršový přesah - jeden verš neodpovídá jedné větě
            \item psáno tak, aby se to dobře četlo do rytmu jazzové hudby - 1 verš na jeden nádech
        \end{itemize}
        \item\textbf{jazykové prostředky a jejich funkce ve výňatku}
        \begin{itemize}
            \item spisovný i nespisovný jazyk
            \item vulgarismy - mrdat do prdele
        \end{itemize}
        \item\textbf{tropy a figury a jejich funkce ve výňatku}
        \begin{itemize}
            \item anafora - kteří
            \item synekdocha - nejlepší hlavy
            \item metafora - hvězdné dynamo ve strojovně noci (osud), okna lebky (oči), 
        \end{itemize}
    \end{itemize}
    \section*{Literárně historický kontext a autor}
    Allen Ginsberg byl syn židovské ruské imigrantky do USA. Studoval Columbijskou universitu - měl problém s drogami a krádežemi, takže ho vyhodili, ale potkal zde Jacka Kerouaca a další beatníky.
    Cestoval do spoustu zemí, experimentoval s drogami, byl homosexuál, v Americe považován za zvrhlíka. Jeho tvorbu ovlivnil surrealismus, dadaismus, buddhismus a existencialismus.
    Kvílení je jeho první důležité dílo a zároveň jeho nejúspěšnější, je často považováno za manifest beat generation. Další díla: Kadiš (věnováno zesnulým rodičům).
    Tvořil v 50. a 60. letech 20. století v době tzv. Sanfranciské renesance, je jedním ze zakládajících členů beat generation. V roce 1956 díky Škvoreckému navštívil Prahu a stal se zde králem Majálesu, měl zde problémy s StB a byl vyhoštěn.
    Má blízko k Jacku Kerouacovi (Na cestě) a Williamu Burroughsovi (nahý oběd) (oba beat generation). Mezi jeho přáteli mělo úspěch, u nás také, ale v Americe byl nakladatel Ferlinghetti souzen, protože dílo bylo považováno za zvrhlé a nemorální. O soudním sporu je natočen film.
\end{document}