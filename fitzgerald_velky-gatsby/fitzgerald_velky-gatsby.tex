\documentclass[11pt]{article}
\usepackage[a4paper, total={18cm, 27cm}]{geometry}
\usepackage[czech]{babel}
\usepackage[utf8]{inputenc}

\begin{document}
    \begin{center}
        \underline{\textbf{\Huge Francis Scott Fitzgerald - Velký Gatsby}}
    \end{center}
    \section*{Úryvek}
    Jsem přesvědčen, že když jsem šel ke Gatsbymu
poprvé, byl jsem jedním z mála hostů, kteří byli skutečně
pozváni. Lidé nebývali zváni – chodili tam. Vsedli do
aut, která je odvezla na Long Island, a tak nebo onak
skončili u Gatsbyho dveří. Když se tam jednou dostali,
někdo, kdo Gatsbyho znal, je představil a potom už se
chovali podle pravidel chování v zábavním parku. Někdy
přišli a odešli, a ani se s Gatsbym nesetkali, přišli na
večírek a prostoduchostí, jež byla sama o sobě
vstupenkou.
Já jsem byl opravdu pozván. Šofér v uniformě
modré jako vajíčko červenky přešel zrána oné soboty
přes můj trávník s překvapivě oficiální pozvánkou od
svého zaměstnavatele: potěšení bude zcela na Gatsbyho
straně, stálo na ní, zúčastním-li se večer jeho „malé
společnosti“. Viděl prý mě několikrát a chtěl mě už
dávno navštívit, ale shoda okolností tomu zabránila –
podepsáno Jay Gatsby, majestátním rukopisem.
Přešel jsem v bílých flanelových kalhotách na jeho
trávník něco po sedmé a potloukal jsem se tam dosti
nesvůj ve vírech a vlnobití lidí, které jsem neznal –
přestože se tu a tam objevila tvář, jíž jsem si už všiml ve
vlaku, který obstarával spojení mezi městem a Západním
Vejcem. Hned jak jsem vstoupil, ohromilo mě, jaká
spousta mladých Angličanů je tu roztroušena všude
kolem; všichni byli dobře oblečeni, všichni vypadali
trochu hladově a všichni mluvili tichými, vážnými hlasy
k solidním, prosperujícím Američanům. Byl jsem si jist,
že něco prodávají; cenné papíry nebo pojistky nebo auta.
Byli si alespoň mučivě vědomi blízkosti snadného zisku
a chovali přesvědčení, že bude jejich, podaří-li se jim
pronést několik slov ve správné tónině.
Jakmile jsem přišel, pokusil jsem se najít svého
hostitele, ale dva nebo tři lidé, kterých jsem se ptal, kde
by asi mohl být, na mne civěli s takovým úžasem a tak
prudce popírali, že by měli nějaké ponětí o tom, kde se
pohybuje, že jsem se odplížil směrem ke koktajlovému
stolu – jedinému místu na zahradě, kde mohl osamocený
člověk prodlévat, aniž by vypadal bezúčelně a sám.
Málem bych se byl už jen z rozpaků pořádně opil,
když vtom vyšla z domu Jordan Bakerová a stanula na
horním konci mramorových schodů, lehce se opřela zády
o zábradlí a pohlédla s přezíravým zájmem dolů do
zahrady.
Ať jí to bylo vhod nebo ne, shledal jsem, že je nutné,
abych se k někomu připojil, než začnu srdečnými
poznámkami oslovovat kolemjdoucí.
    \section*{Analýza uměleckého textu}
    \begin{itemize}
        \item\textbf{zasazení výňatku do kontextu díla}
        \begin{itemize}
            \item začátek knihy, hlavní postava poprvé navštíví Gatsbyho párty po přestěhování se na Long Island, chvíli po této ukázce se poprvé seznámí s Gatsbym
        \end{itemize}
        \item\textbf{téma a motiv}
        \begin{itemize}
            \item\textbf{téma:} kritika zkaženosti americké společnosti 20. let 20. století, kritika amerického snu - všichni jdou pouze za bohatstvím, oslavami a zážitky a nezáleží jim na ostatních lidech (hezky vidět po Gatsbyho smrti, kdy mu na pohřeb nepřijde nikdo z jeho bývalých hostů, protože ho často ani neznali), bohatým projdou věci, které chudým ne
            \item\textbf{motivy: }prohibice, alkohol, nevěra, bohatství (zděděné x získané), počasí (podle dané situace - déšť, \dots)
        \end{itemize}
        \item\textbf{časoprostor}
        \begin{itemize}
            \item\textbf{prostor: }reálný, konkrétní - Západní vejce, Long Island, New York
            \begin{itemize}
                \item bydlí zde bohatí lidé ve svých vilách - oddělené od zbytku města, pořádají okázalé večírky - tak to asi tehdy bylo
            \end{itemize}
            \item\textbf{čas: }20. léta 20. století - Jazzový věk - vznikají jazzové kluby a jazz obecně se dostává do poularity, doba amerického snu - imigranti z jiných zemí přichází hledat bohatství
        \end{itemize}
        \item\textbf{kompoziční výstavba}
        \begin{itemize}
            \item dílo je rozděleno do 9 kapitol
            \item před první kapitolou je krátká báseň od Thomase Parke D'Invilliers
            \item děj chronologický, jsou zde krátké retrospektivní pasáže, které objasňují Gatsbyho minulost (jak přišel ke svému bohatství, \dots)
        \end{itemize}
        \item\textbf{literární žánr a druh}
        \begin{itemize}
            \item\textbf{druh: }epika, próza
            \item\textbf{žánr: }román (společenský)
        \end{itemize}
        \item\textbf{vypravěč / lyrický subjekt}
        \begin{itemize}
            \item vypravěčem je jedna z hlavních postav - Nick Carraway
            \item osobní vypravěč - nedává nám všechny informace, podává příběh ze svého pohledu
            \item ich-forma
        \end{itemize}
        \item\textbf{postava}
        \begin{itemize}
            \item Nick Carraway - hlavní (vypravěč), fiktivní, přírozená - přijel do New Yorku vydělat peníze na burzu, vypráví ze svého pohledu, ale je většinou v pozadí dění, diví se způsobům bohatých newyorčanů - chodí ke Gatsbymu bez pozvání, jediný Gatsbyho opravdový přítel \dots
            \item Jay Gatsby - hlavní postava, fiktivní, přirozená, ztělesnění amerického snu (byl chudý, ale teď je bohatý svým přičiněním), dalo by se říct, že je hypotetická - nevíme, jestli je kladná či záporná, je tajemná, nevíme skoro nic, jenom vymyšlené historky, které kolují mezi ostatními a ti ho většinou ani neviděli; postupně se dozvídáme, že se jmenuje James Gatz a že veškeré jmění pašováním alkoholu (je prohibice). Jeho cílem, za kterým si velmi silně jde, je získat Daisy, Nickovu sestřenici, se kterou se seznámil před lety a která je provdaná za Toma Buchanana, nakonec ho zastřelí manžel ženy, kterou Daisy omylem přejede, když jela s Gatsbym autem a se kterou ji Tom podváděl.
            \item Tom Buchanan - vedlejší postava, fiktivní, přirozená - představuje člověka, který své bohatství zdědil; manžel Daisy, podvádí jí s Myrtle, kterou pak Daisy omylem zabije. Vznětlivý, agresivní, namyšlený.
            \item Daisy - hlavní postava, fiktivní přirozená - krásná, ale sobecká, ublíží i Gatsbymu i Tomovi - když se o ní hádají, nejdříve Gatsbymu říká, že Toma nikdy nemilovala, ale pak před ním to odmítne přiznat
        \end{itemize}
        \item\textbf{vyprávěcí způsoby}
        \begin{itemize}
            \item ich-forma z pohledu hlavní postavy
        \end{itemize}
        \item\textbf{typy promluv}
        \begin{itemize}
            \item přímá (není v úryvku) a nepřímá řeč
        \end{itemize}
        \item\textbf{jazykové prostředky a jejich funkce ve výňatku}
        \begin{itemize}
            \item spisovný jazyk
            \item celkem dlouhá souvětí, hlavně v rozsáhlých popisech Gatsbyho bohatství nebo jeho párty, v promluvách postav ne tak dlouhá
        \end{itemize}
        \item\textbf{tropy a figury a jejich funkce ve výňatku}
        \begin{itemize}
            \item Gatsby pořád opakuje na konci oslovení "kamaráde" ??
            \item přirovnání - uniforma modrá jako vajíčko červenky
            \item moc jich v tomto díle není
        \end{itemize}
    \end{itemize}
    \section*{Literárně historický kontext a autor}
    Americký spisovatel, syn irských přistěhovalců, člen ztracené generace (generace amerických autorů, kteří žili v Paříži a byli poznamenáni WWI).
    Žil jako celebrita, ale jen jeho první román měl úspěch. Aby si mohl udržet životní styl, psal do novin a prodával svá díla do Hollywoodu.
    Vzal si za manželku Zeldu, ze které se později stala schizofrenička. Kvůli velkým výdajům na léčbu a na životní styl se zadlužil.
    Později se stal alkoholikem, musel se živit jako Hollywoodský scénárista.
    Tvořil v meziválečném období (mezi WWI a WWII).
    Velký Gatsby je jeho 3. román.
    Dílo bylo několikrát zfilmováno, zdivadelněno, dokonce i balet.
    Kromě Velkého Gatsbyho napsal ještě Na prahu ráje, Krásní a prokletí, Poslední magnát (nedokončen, vydán posmrtně) a Povídky jazzového věku (povídková sbírka).
    Dobře přijatý byl pouze Na prahu ráje, ostatní se proslavily až posmrtně - v ostatních dílech kritizoval společnost \dots
    Ve stejné době tvořil např. Ernest Hemingway, Eliot, Steinbeck nebo Remarque.
\end{document}