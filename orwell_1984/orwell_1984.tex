\documentclass[11pt]{article}
\usepackage[a4paper, total={18cm, 27cm}]{geometry}
\usepackage[czech]{babel}
\usepackage[utf8]{inputenc}

\begin{document}
    \begin{center}
        \underline{\textbf{\Huge George Orwell - 1984}}
    \end{center}
    \section*{Úryvek}
    Za Winstonovými zády stále ještě někdo žvanil z obrazovky o šedé litině a o překročení Deváté tříletky. Obrazovka současně přijímala a vysílala. Každý zvuk, který Winston vydal a jenž byl hlasitější než velmi tiché šeptání, obrazovka zachycovala; a co víc, pokud zůstával v zorném poli kovové desky, bylo ho vidět a slyšet. Samozřejmě, člověk si nikdy nebyl jist, zda ho v daném okamžiku sledují. Jak často a podle jakého systému Ideopolicie zapínala jednotlivá zařízení, bylo hádankou. Předpokládalo se, že sledují každého neustále. A rozhodně mohli zapnout vaše zařízení, kdy se jim chtělo. Člověk musel žít – a žil, ve zvyku, který se stal pudovým, – v předpokladu, že každý zvuk, který vydá, je zaslechnut, a každý pohyb, pokud není tma, zaznamenán.
Winston zůstal obrácen zády k obrazovce. To bylo bezpečnější; ačkoli, jak dobře věděl, i záda mohou ledacos prozradit. Kilometr odtud se tyčila nad špinavou krajinou vysoká bílá budova Ministerstva pravdy, jeho pracoviště. Toto je Londýn, pomyslel si s jistou nechutí, hlavní město Územní oblasti jedna, třetí nejlidnatější provincie Oceánie. Snažil se vydolovat nějakou vzpomínku z dětství, která by mu řekla, zda Londýn býval vždycky takový. Byla tu odjakživa tahle vyhlídka na rozpadající se domy z devatenáctého století, podepřené z boku dřevěnými trámy, s okny zatlučenými překližkou a střechami z rezavého plechu, těmi vetchými zahradními zdmi bortícími se na všech stranách? Byly tu odjakživa trosky po bombardování, nad nimiž ve vzduchu víří prach z omítky, i vrby sklánějící se nad hromadami suti? A prostranství, kde bomby vymýtily větší plochu, na níž pak vyrazili ubohé kolonie dřevěných chatrčí, podobných kurníkům? Ale nemělo to smysl, nemohl si vzpomenout: z dětství mu v paměti nezůstalo nic než pár jasně osvětlených obrázků, které neměly žádné pozadí a byly většinou nesrozumitelné.
Ministerstvo pravdy, v newspeaku Pramini (Newspeak byl úředním jazykem Oceánie. O její struktuře a etymologii viz Dodatek.), se děsivě lišilo od všech ostatních objektů v dohledu. Byla to obrovská stavba tvaru pyramidy ze zářivě bílého betonu, která se terasovitě vypínala do výšky 300 metrů. Z místa, kde stál Winston, se dala na bílém průčelí přečíst ozdobným písmem vyvedená tři hesla Strany:

VÁLKA JE MÍR

SVOBODA JE OTROCTVÍ

NEVĚDOMOST JE SÍLA
    \section*{Analýza uměleckého textu}
    \begin{itemize}
        \item\textbf{zasazení výňatku do kontextu díla}
        \begin{itemize}
            \item začátek díla, skrze hlavního hrdinu se dozvídáme základy o prostředí a době, kdy se děj odehrává
        \end{itemize}
        \item\textbf{téma a motiv}
        \begin{itemize}
            \item \textbf{téma - }neustálé sledování, manipulace, strach, fungování totalitních režimů
            \item \textbf{motivy - }udržovací válka, thoughtcrime, 2 minuty nenávisti, dělení obyvatel na třídy, nedostatek komodit, "vítězství" (leitmotiv, cigarety, gin, sídliště, \dots)
        \end{itemize}
        \item\textbf{časoprostor}
        \begin{itemize}
            \item \textbf{čas - }budoucnost (rok 1984) - autorova předtucha, jak by mohla vypadat společnost ve fiktivní budoucnosti
            \item \textbf{prostor - }Londýn, jedno z měst Oceánie, jednoho ze tří zbývajících mocností na světě (sousední s Eurasií a Eastasií) - opět předpověď do budoucna, co by se mohlo se světem stát
        \end{itemize}
        \item\textbf{kompoziční výstavba}
        \begin{itemize}
            \item Členěno do tří částí - podle vývoje postavy
            \begin{itemize}
                \item první část ví, že je něco špatně, ale nevzdoruje proti straně;
                \item ve druhé části se seznámí s Julií a postupně dělají více a více přestupků proti ideologii, až si nakonec pronajmou pokoj v prolétské čtvrti a čtou knihu, která údajně vysvětluje princip fungování totality;
                \item třetí část obsahuje děj na ministerstvu lásky, kde Winstona mučí a přetvářejí ho jako člověka, aby věřil ideologii
                \item na konci je doslov - autor vysvětluje původ a detaily o \textit{newspeaku} (odosobněný jazyk bez emocí, nelze v něm vyjádřit žádné protiideologické myšlenky)
            \end{itemize}
            \item dále členěno do kapitol - podobně dlouhé, druhá část nejdelší
            \item časová kompozice chronologická
        \end{itemize}
        \item\textbf{literární žánr a druh}
        \begin{itemize}
            \item \textbf{druh: }epika, próza
            \item \textbf{žánr: }román
        \end{itemize}
        \item\textbf{vypravěč}
        \begin{itemize}
            \item nadosobní vševědoucí, nevyjadřuje názor, jen popisuje děj
            \item er-forma
        \end{itemize}
        \item\textbf{postava}
        \begin{itemize}
            \item Winston Smith - hlavní, fiktivní, přirozená; vývojová postava - postupně se odhodlává k více a více přestupkům proti zákonu, po zatčení a mučení se z něj naopak stává prázdný člověk, který už nemá žádné revoluční myšlenky a naopak miluje Stranu a Velkého bratra
            \item Julie - vedlejší, fiktivní, přirozená; spíše statická - už před Winstonem měla několik ilegálních milenců, po zatčení je stejná jako Winston a jejich láska mučením zmizela
            \item Velký bratr - vedlejší, fiktivní, nadpřirozená - nevíme o ní nic, ani jestli existuje; je to tvář a modla Strany, na plakátech, ale jinak v ději nevystupuje
            \item Emanuel Goldstein - vedlejší, fiktivní, nadpřirozená - také o ní nic nevíme, je protikladem Velkého bratra - ztělesňuje opozici proti Straně, nejčastěji se objevuje ve 2 minutách nenávisti jako osoba, která může za všechno zlé, prý je vůdcem Bratrstva.
            \item O'Brien - vedlejší, fiktivní, přirozená - jeden z vrchních hodnostářů Strany, oklame Winstona s Julií jako vůdce podsvětní organizace Bratrstvo, po celou dobu je sleduje a čeká na důkazy k zatčení, pak je mučí na ministerstvu lásky
        \end{itemize}
        \item\textbf{vyprávěcí způsoby}
        \begin{itemize}
            \item er-forma
        \end{itemize}
        \item\textbf{typy promluv}
        \begin{itemize}
            \item přímá a nepřímá řeč - ne v ukázce, přímá v uvozovkách
        \end{itemize}
        \item\textbf{jazykové prostředky a jejich funkce ve výňatku}
        \begin{itemize}
            \item spisovný jazyk
            \item neologismy - některé výrazy jsou v autorově vymyšleném jazyce - \textit{newspeak} - v ukázce Ideopolicie, Pramini, dále vaporizace, doublethink
            \begin{itemize}
                \item newspeak je vymyšlený úřední jazyk Oceánie, který je osvobozen od veškerých emocí a víceznačnosti, lze jím vyjádřit pouze myšlenky v souladu s ideologií
            \end{itemize}
            \item věty v dějových částech jednoduché nebo krátká souvětí - akčnější plynutí děje
            \item ve filozofických pasážích (především kniha vysvětlující fungování režimu) jsou spíše delší souvětí - je to vlastně úvaha nebo zamyšlení bez děje - vyjadřuje složité myšlenky
        \end{itemize}
        \item\textbf{tropy a figury a jejich funkce ve výňatku}
        \begin{itemize}
            \item jazyk není moc bohatý na figury ani tropy
            \item oxymoron - válka je mír, svoboda je otroctví - ironicky ukazuje nesmyslnost totalitních režimů
            \item symbol - Velký bratr - symbol manipulace, nevíme o něm nic - způsobuje to tajemno
        \end{itemize}
    \end{itemize}
    \section*{Literárně historický kontext a autor}
    George Orwell (Eric Arthur Blair) byl významný anglický autor antiutopií. Narodil se v Indii, studoval v Anglii, pak se vrátil do Barmy pracovat jako policista.
    Byl i významným novinářem a esejistou. Kritizoval imperialismus a kapitalismus, ale i totalitní režimy v Rusky, za WWII pracoval pro BBC. Velmi známý svou analýzou fungování totalitních režimů.
    Tvořil především za a po WWII.
    Spoustu věcí předpověděl do budoucna, např. sledování veřejného prostoru, televizní obrazovky, NSA v USA. Proslavil se hlavně svým dílem Farma zvířat, což je alegorie (bajka v próze) na komunistický režim, a tímto dílem.
    U nás zakázáno, vycházelo v samizdatu, v západních zemích kriticky dobře přijato. Spoustu citátů z knihy přešlo do obecné znalosti - Velký bratr tě sleduje.
    Z autorů tvořící podobný žánr je např. Ray Bradbury a především jeho dílo 451 stupňů Fahrenheita, nebo Marťanská kronika.
\end{document}