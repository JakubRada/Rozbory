\documentclass[11pt]{article}
\usepackage[a4paper, total={18cm, 27cm}]{geometry}
\usepackage[czech]{babel}
\usepackage[utf8]{inputenc}

\begin{document}
    \begin{center}
        \underline{\textbf{\Huge J.D.Salinger - Kdo chytá v žitě}}
    \end{center}
    \section*{Úryvek}
    Pro případ, že nebydlíte v New Yorku, tak Wicker bar je v jednom takovým
hogo fogo hotelu, v hotelu Seton. Já jsem tam dost často chodíval, ale už
nechodím. Postupem času jsem toho úplně nechal. Je to takovej ten
podnik, kam se chodí, aby se o člověku říkalo, že je ohromně kultivovanej,
a tajtrlíkama se to tam jen hemží. Mívali tam dvě francouzský slečny – Tinu
a Janinu – který tam tak třikrát za večer vylezly a zahrály na piano a
zazpívaly. Jedna z nich hrála na piano – vyloženě mizerně – a ta druhá
zpívala a většina šlágrů byla pěkně sprostá, anebo to bylo francouzsky. Ta,
co zpívala, Janina, vždycky než začala zpívat, řekla pár slov do mikrofonu.
Vždycky zaševelila: „A nyný bykom vám ktěli pržedvesty Vulé vu francé. Je
to istorie maličké francouzské dyvenky, co pržide do velkého mjesta jako
New York a zamiluje se do malého klapce z Brooklynu. Doufáme, že ona
vám bude líbity.“ Načež když si to odševelí – a přestane si hrát na
roztomilou – zazpívá nějakej nesmyslnej song napůl anglicky a napůl
francouzsky a všichni tajtrlíci v lokále se div nepotentočkujou radostí.
Kdybyste tam zůstali sedět dost dlouho a poslouchali, jak všichni ti šašci
tleskají, začli byste nenávidět všechny lidi na světě, na mou duši, že byste
začali.
A ten jejich barman byl taky pěknej prevít. Náramnej snob. Skoro s
člověkem ani nepromluvil, leda když jste byli velký zvíře nebo nějaké
hvězda nebo něco. Ale kdybyste byli velký zvíře nebo nějaká hvězda nebo
něco, tak by se vám z něj udělalo teprve nanic. To by k vám přišel s
takovým tím grandiózním, okouzlujícím úsměvem, jako by byl bůhvíjak
prima chlap, když ho člověk pozná blíž, a řek by: „Tak co je novýho v
Connecticutu?“ nebo „Co je novýho na Floridě?“ Byl to příšernej podnik.
Bez legrace. Postupem času jsem tam úplně přestal chodit.
Když jsem tam přišel, bylo ještě hrozně brzo, pročež jsem se posadil k
baru – byl tam pěknej nášvih – a udělal jsem pár whisky se sodou, ještě
než se Luče vůbec vynořil. Když jsem si je objednával, tak jsem se postavil,
aby bylo vidět, že jsem velkej, a aby si nemysleli, že jsem nějakej čápek.
Pak jsem chvilku pozoroval ty tajtrlíky. Jeden chlápek hned vedle mě dělal
do jedny slečny, co měl s sebou. V jednom kuse jí říkal, že má
aristokratický ruce. To mě umrtvilo. Na druhým konci baru bylo plno
teplajzníků. Ne že by vypadali nějak moc teplajznicky – jako že neměli moc
dlouhý vlasy a tak – ale přesto bylo vidět, že jsou teplí. Konečně se vynořil
Luče.
    \section*{Analýza uměleckého textu}
    \begin{itemize}
        \item\textbf{zasazení výňatku do kontextu díla}
        \begin{itemize}
            \item 2. polovina knihy, kdy už je Holden nějakou chvíli v New Yorku
            \item chvíli po této scéně se potají vrátí k nim domů a povídá s Phoebe, rodiče o tom neví
        \end{itemize}
        \item\textbf{téma a motiv}
        \begin{itemize}
            \item\textbf{téma: }dospívání (konfrontace dětských ideálů s životem dospělých), hledání identity (sledujeme mladého chlapce, který neví, co bude v životě dělat), ztráta iluzí
            \item\textbf{motivy: }nepochopení, pocit osamocení, pocit nejistoty, smrt (mladší bratr Allie zemřel na leukemii, Holdenovi hodně chybí)
        \end{itemize}
        \item\textbf{časoprostor}
        \begin{itemize}
            \item\textbf{čas: }50. léta 20. století, zima - autorova současnost, problém s hledáním identity je větší než kdy dřív (nejednou je více možností, nejistota)
            \item\textbf{prostor: }reálný, konkrétní svět - internátní škola poblíž New Yorku, New York; autorovo rodné město, zmatek velkoměsta může ztěžovat hledání identity
        \end{itemize}
        \item\textbf{kompoziční výstavba}
        \begin{itemize}
            \item členěna na 26 kapitol
            \item všechny podobně dlouhé, až na poslední kapitolu, která je výrazně kratší a Holden v ní zmiňuje, že byl v léčebně a že se vrátí do školy
            \item retrospektivní vyprávění z blázince, představuje nám události minulého prosince; samotné vyprávění je chronologické
        \end{itemize}
        \item\textbf{literární žánr a druh}
        \begin{itemize}
            \item\textbf{druh: }epika, próza
            \item\textbf{žánr: }román (psychologický, o dospívání)
        \end{itemize}
        \item\textbf{vypravěč / lyrický subjekt}
        \begin{itemize}
            \item osobní vypravěč - sám Holden je vypravěčem
            \item nespolehlivý vypravěč - představuje nám jenom co chce a nevíme jestli je to pravda
            \item působí to, jako by to vyprávěl svému kamarádovi, opakuje stejné fráze
            \item ich-forma
        \end{itemize}
        \item\textbf{postava}
        \begin{itemize}
            \item Holden Claufield - vypravěč, hlavní postava, fiktivní, přirozená, neutrální - 17 letý chlapec, už byl na několika školách, ze všech ho vyhodili, dobré známky má jenom z angličtiny, z ostatních propadá, má šedivé vlasy (je jiný než ostatní), je velmi citlivý, přemýšlí o věcech do hloubky - maskuje to, říká, že mu je všechno jedno, cítí se nepochopeně a osamoceně, neví co bude dělat v životě, snaží se mluvit s někým, kdo by mu porozuměl, jediný s kým si rozumí je Phoebe a mrtvý mladší bratr, staršího bratra má taky rád, ale kritizuje ho, že se dal k filmu
            \item Phoebe - vedlejší postava, fiktivní, přirozená - jeho mladší sestra, je dítě a je ještě nezkažená - proto ji má tolik rád
        \end{itemize}
        \item\textbf{vyprávěcí způsoby}
        \begin{itemize}
            \item id-forma
            \item působí to, jako by to vyprávěl svému kamarádovi, opakuje stejné fráze
        \end{itemize}
        \item\textbf{typy promluv}
        \begin{itemize}
            \item přímá řeč, nepřímá řeč (V jednom kuse jí říkal, že má aristokratický ruce)
        \end{itemize}
        \item\textbf{jazykové prostředky a jejich funkce ve výňatku}
        \begin{itemize}
            \item převážně nespisovný jazyk
            \item obecná čeština - kultivovanej, pěknej prevít
            \item slang - teplajzník, čápek
            \item některá slova působí zastarale - způsobeno překladem zastaralého anglického slangu, který nemá moc dobré české ekvivalenty + překládáno docela dávno - některá slova dnes zní divně (nepotentočkujou, tajtrlíci, )
            \item dlouhá souvětí v monologu vypravěče - občas anakolut - možná - (začne větu o něčem a skončí něčím jiným) - lépe to představuje přemýšlení
            \item často opakuje některé fráze (to by mě umrtvilo, deprimovalo mě to, \dots)
        \end{itemize}
        \item\textbf{tropy a figury a jejich funkce ve výňatku}
        \begin{itemize}
            \item metafora - velký zvíře, hvězda, aritstokratický ruce
            \item oxymoron - pěknej prevít ??
        \end{itemize}
    \end{itemize}
    \section*{Literárně historický kontext a autor}
    J.D.Salinger byl americký autor, účastnil se WWII.
    Jeho díla z meziválečného období mu nikdo nevydával.
    Po válce napsal svůj první román Kdo chytá v žitě, který zaznamenal obrovský úspěch, na jihu byl zakazován kvůli vulgaritě, což ho učinilo ještě populárnějším.
    Nevyhledával úspěch a žádné z jeho dalších děl už ho nemělo. Napsal ještě pár povídek (Devět povídek, Franny a Zooey), ale pak přestal psát a žil jen z autorských práv.
    Jakoukoli filmovou adaptaci jeho díla zakázal.
    Je spojováno s vraždou Johna Lenona - jeho vrah měl tuto knihu u sebe, do které napsal poslední zprávu (To Holden, From Holden, This is my statement)
    Přátelil se s Hemingwayem, ve stejné době tvořili např. Golding (Pán much), Bradbury, Lem, Beckett, Ionesco (absurdní drama) u nás např. Škvorecký (Zbabělci).
\end{document}