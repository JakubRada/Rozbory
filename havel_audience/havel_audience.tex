\documentclass[11pt]{article}
\usepackage[a4paper, total={18cm, 27cm}]{geometry}
\usepackage[czech]{babel}
\usepackage[utf8]{inputenc}

\begin{document}
    \begin{center}
        \underline{\textbf{\Huge Václav Havel - Audience}}
    \end{center}
    \section*{Úryvek}
    SLÁDEK Co byste říkal tomu dělat tady skladníka?To by  nebylošpatný,  co?  Jstekoneckoncůinteligent, poctivej  jste  taky,  tak co? Nebudete  přece  pořád jen  koulet  scikánama  sudy!  Byl  byste  v  teple–přes  poledne  byste  to  zavřel–jakože  dělátepořádek–a  moh byste  si  tam v klidu  vymejšlet  nějakývtipy do těch  svejch her–akdybyste chtěl, mohbyste si tam třeba i zdřímnout–co tomuříkáte?\\ VANĚKMyslíte,že by to bylo možné?\\ SLÁDEKCo by to nebylo možný?\\ VANĚKJá nejsem samozřejměv situaci,že bychsimohl vybírat, ale pokud by tu takovámožnostopravdu  byla,  považoval  bych  to  samozřejmězavýborné–smysl  propořádek, myslím, mám–nastroji psát umím–trochu znám i cizí jazyky–tovíte, vtom sklepěje přece jen dost zima–zvlášťkdyžna točlověk není zvyklý–\\ SLÁDEKNo právě. Rozumíte evidenci?\\ VANĚKUrčitěbych to pochopil–mámčtyři semestry ekonomie–\\ SLÁDEKJo? A rozumíte evidenci?\\ VANĚKUrčitěbych to pochopil–\\ SLÁDEKByl  byste  v  teple–přes  poledne  byste  tozavřel–nebudete  přece  pořád  jenkoulet s cikánama sudy!\\ VANĚKPokud by tu taková možnost byla–(Pauza)\\ SLÁDEKNe,  ne,  Vaňku,  kdyžje  někdo  křivák,  poznám  to  užna  dálku!  Vy  jste  poctivejchlap,  jájsem  taky  poctivej,  tak  nevím,  pročbychom  to  nemohli  táhnout  spolu!  Coříkáte?\\ VANĚKAno–jistě–\\ SLÁDEKJste teda pro?\\ VANĚKSamozřejmě–\\ SLÁDEKJestli nechcete, tak tořekněte! Třeba sevám se mnou do party nechce–třebamáte protimněvýhrady–třeba máte jiný plány–\\ VANĚKNemám  proti  vám  výhrady–naopak–udělal  jste  pro  měhodně–jsem  vámvelice  zavázán–zvlášťkdyby  to  vyšlo  s  tím  skladem–chci  přirozeněudělat  vše,abyste mohl být i vy s mou pracíspokojen–(\\ SLÁDEK otevře další láhev a naleje Vaňkovi i sobě)\\ SLÁDEKTak to můžem zapít?\\ VANĚKAno–(Oba pijí)\\ SLÁDEKNa ex!(\\ VANĚK s obtížemi dopije sklenku, \\ SLÁDEK mu okamžitěopět naleje. Pauza)A nebuďte smutnej!\\ VANĚKJá nejsem smutný–(Pauza)\\ SLÁDEKTy, Ferdinande–\\ VANĚKAno?\\ SLÁDEKJsme kamarádi, jo?\\ VANĚKAno–\\ SLÁDEKNeříkášto jen tak?\\ VANĚKNe–\\ SLÁDEKVěříšmi teda?\\ VANĚKPřirozeně,že vám věřím–\\ SLÁDEKPočkej, ale upřímně: věříšmi?\\ VANĚKVěřím vám–\\ SLÁDEKTak hele–něco ti povím–ale to ječistěmezi náma, jasný?
    \\ VANĚKJasný–\\ SLÁDEKMůžu se spolehnout?\\ VANĚK Můžete–\\ SLÁDEK Tak hele–(Ztiší hlas)Chodějí se sem natebe ptát–\\ VANĚK Rdo?\\ SLÁDEK No přece oni–\\ VANĚK Vážně?\\ SLÁDEK Fakt–\\ VANĚK  A  máte  dojem–pokud  jde  o  mé  zaměstnánítady  v  pivovaře–že  je  nějakohroženo?(Pauza)Nenaléhají,  abych  byl  propuštěn?(Pauza)Anebonevyčítají  vám,že jste měsem vzal?(Pauza)\\ SLÁDEKTak hele–něco ti povím–ale to ječistěmezi náma, jasný?\\ VANĚK Jasný–\\ SLÁDEK Můžu se spolehnout?\\ VANĚK Můžete–\\ SLÁDEK Tak hele–sedět na tomhle místěněkdojinej nežjá, taku nás neděláš, za to titeda ručím!Stačí ti to?\\ VANĚK Ano, jistě–jsem vám velmi vděčen–\\ SLÁDEK Neříkám to, abys mi děkoval–\\ VANĚK Já vím,že ne–\\ SLÁDEK Já jen, abys věděl, jak stojí situace–\\ VANĚK Děkuji vám–
    \section*{Analýza uměleckého textu}
    \begin{itemize}
        \item\textbf{zasazení výňatku do kontextu díla}
        \begin{itemize}
            \item z druhé poloviny knihy, kdy sládek začíná Vaňkovi vysvětlovat, že se na něj chodí ptát lidé ze strany
            \item chvíli poté ho požádá, aby na sebe sám donášel
        \end{itemize}
        \item\textbf{téma a motiv}
        \begin{itemize}
            \item\textbf{téma: }absurdita morálky - idiot nařizuje intelektuálovi, zobrazení intelektuála perzekuovaného systémem
            \item\textbf{motivy: }rozdíl mezi intelektuály a nižší vrstvou - jazyk, práce v pivovaru
        \end{itemize}
        \item\textbf{časoprostor}
        \begin{itemize}
            \item\textbf{čas: }70. léta 20. století - autorova současnost, období totality
            \item\textbf{prostor: }sládkova kancelář, nejmenovaný pivovar
        \end{itemize}
        \item\textbf{kompoziční výstavba}
        \begin{itemize}
            \item jednoaktová hra, velmi krátká
            \item chronologická
            \item stále se opakující fráze a dialogy, na konci monolog
        \end{itemize}
        \item\textbf{literární žánr a druh}
        \begin{itemize}
            \item\textbf{druh: }drama
            \item\textbf{žánr: }absurdní drama, tragikomedie ?
        \end{itemize}
        \item\textbf{vypravěč / lyrický subjekt}
        \begin{itemize}
            \item vypravěč není, je to drama, pouze scénické poznámky
        \end{itemize}
        \item\textbf{postava}
        \begin{itemize}
            \item Vaněk - autobiografická postava, hlavní, kladná - slušný intelektuál, disident, pracuje v pivovaru zatímco píše divadelní hry, uctivý, zásadový
            \item Sládek - hlavní, fiktivní, přirozená, záporná - neslušný, vulgární buran, pokrytecký - snaží se neznelíbit mocným, čím dál více se opíjí, ředitel pivovaru, nabízí Vaňkovi lepší pozici a chce, aby na sebe donášel a nemusel to on řešit, nakonec si mu stěžuje, jak mají intelektuálové jednoduchý život na rozdíl od ostatních
        \end{itemize}
        \item\textbf{vyprávěcí způsoby}
        \begin{itemize}
            \item du-forma, dialogy, monolog na konci
        \end{itemize}
        \item\textbf{typy promluv}
        \begin{itemize}
            \item repliky, dialogy, monolog na konci
        \end{itemize}
        \item\textbf{jazykové prostředky a jejich funkce ve výňatku}
        \begin{itemize}
            \item Vaněk spisovný jazyk, vyká, uctivý
            \item Sládek - nespisovný jazyk, tyká, vulgární, neslušný, skáče Vaňkovi do řeči
            \begin{itemize}
                \item to by nebylo špatný - obecná čeština
                \item a nebuďte smutnej
                \item oni se neposerou - vulgarismus
            \end{itemize}
            \item ... kontrast mezi intelektuálem a buranem
            \item občas docela dlouhé věty, hlavně sládek
        \end{itemize}
        \item\textbf{tropy a figury a jejich funkce ve výňatku}
        \begin{itemize}
            \item táhnout spolu - metafora ??
            \item jsem vám zavázán - metafora
            \item dopije sklenku - metonymie
        \end{itemize}
    \end{itemize}
    \section*{Literárně historický kontext a autor}
    Václav Havel, poslední prezident Československa a první prezident České republiky, byl světově proslulý dramatik a významný esejista.
    Známý po celém světě, několikrát nominovaný na Nobelovu cenu míru.
    Dostudoval až v 60. letech - byl buržoazního původu.
    V 60. letech tvořil hry kritizující obecně absurditu režimu, byrokratizaci - Zahradní slavnost, Vyrozumění - kritika jazyka, vymyšlený jazyk Ptydepe. Věnoval se i experimentální poezii - Antikódy (typogramy).
    Založil samizdatové nakladatelství Expedice a Chartu 77 (reakce na zatčení skupiny Plastic people of the Universe, porušování lidských práv).
    Změnil svou poetiku - více autobiografické vaňkovské hry - zobrazují člověka intelektuála perzekuovaného režimem - Audience.
    Významný esejista - Dopis Gustavu Husákovi (otevřený dopis, perfektní analýza strachu a přetvářky lidí v režimu), Moc bezmocných.
    Tvořil v 2. polovině 20.století a na začátku 21. století.
    Autoři ze stejného období - Ludvík Vaculík, Hrabal, Hrubín, Klíma, ve světě - Beckett, Ionesco (absurdní drama).
    Je televizní podoba a audionahrávka. Hra měla velký úspěch.
\end{document}