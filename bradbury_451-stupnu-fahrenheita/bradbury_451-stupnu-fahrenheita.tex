\documentclass[11pt]{article}
\usepackage[a4paper, total={18cm, 27cm}]{geometry}
\usepackage[czech]{babel}
\usepackage[utf8]{inputenc}

\begin{document}
    \begin{center}
        \underline{\textbf{\huge Ray Douglas Bradbury - 451 stupňů Fahrenheita}}
    \end{center}
    \section*{Úryvek}
    Po chodníku zalitém měsíčním světlem letělo podzimní listí, takže se zdálo, jako by dívka, která se pohybovala v jeho proudu, ujížděla vpřed a nechávala se pohybem větru a listů unášet. Se skloněnou hlavou pozorovala své střevíce brodící se ve zvířených listech. Měla drobnou, mléčně bílou tvář a v ní výraz duchovního hladu, jenž vše s neúnavnou dychtivostí ohmatával. Zírala na všechno bledá překvapením; tmavé oči hleděly na svět tak soustředěně, že jim žádné hnutí neušlo. Měla bílé šaty a ty šeptaly. Téměř se mu zdálo, že slyší pohyb jejích rukou - a teď mizivě nepatrný zvuk, bělostné zachvění její tváře, kterou k němu obrátila, když zjistila, že blízko ní stojí uprostřed chodníku muž a čeká.
    
    V korunách stromů silně zahučelo, jak se z nich snášel suchý déšť. Dívka zastavila a zdálo se, že snad překvapením ucouvne, ale zatím zůstala stát a prohlížela si Montagna očima tak temnýma a lesklýma a živoucíma, až si pomyslil, že jí asi musel říct něco neobyčejně úžasného. Věděl však, že pohnul rty, jen aby ji pozdravil, a když si všiml, že jí učaroval salalmandr na jeho rukávu a odznak s Fénixem na prsou, promluvil znovu.
    
    "Už vím," řekl, "vy jste naše nová sousedka."
    
    "A vy jste určitě -" pozdvihla oči z odznaků jeho povolání "- požárník." Její hlas se ztratil v dálce.
    
    "Říkáte to nějak divně!"
    
    "Poznala bych - poznala bych to se zavřenýma očima", řekla pomalu.
    
    "Co - zápach petroleje? Moje žena si na něj pořád stěžuje," zasmál se. "Člověk se ho nikdy úplně nezbaví."
    
    "Ne, to nezbaví," řekla stísněně.
    
    Cítil, že kolem něho obchází, převrací ho naruby, jemně jím třese a vyndává mu věci z kapes, aniž jedinkrát pohnula.
    
    "Pro mě," řekl, protože mlčení trvalo už dlouho, "je petrolej jako voňavka."
    
    "Opravdu vám tak připadá?"
    
    "Samozřejmě. Proč by ne?"
    
    Dala si s odpovědí načas, aby o tom mohla přemýšlet. "nevím." Otočila se k chodníku, který vedl k jejich domovům. "Nebude vám vadit, když půjdu s vámi? Já jsem Clarissa McClellanová."
    
    "Clarissa. Já jsem Guy Montag. Jen pojďte se mnou. Co se tu tak v noci potulujete? Kolik vám je?"
    
    Šli po stříbrném chodníku nocí, jíž vál vlahý i chladivý vítr, ve vzduchu slabounce voněly meruňky a jahody a on se rozhlédl kolem sebe a uvědomil si, že v téhle pokročilé roční době to je nemožné.
    
    Byla tu jenom ta dívka. Šla teď s ním, v měsíčním světle svítila její tvář jako sníh a on věděl, že v hlavě převrací jeho otázky a přemýšlí, jak by mu na ně nejlépe odpověděla.
    
    "Je mi sedmnáct a jsem cvok," řekla. "Můj strýček říká, že to jde jedno s druhým. Prý když se tě lidi budou ptát, kolik ti je, pověz jim, že sedmnáct a že jsi cvok. - Že se takhle v noci pěkně jde? Já ráda čichám k věcem a dívám se na ně a někdy zůstanu vzhůru a prochodím celou noc a dívám se na východ slunce."
    
    Šli dál a opět mlčeli, až konečně řekla zamyšleně: "Víte, že se vás vůbec nebojím?"
    
    Překvapilo ho to. "Proč byste se měla bát?"
    
    "Tolik lidí se vás bojí. Myslím vás požárníků. Ale jste vlastně docela normální člověk..."
    
    Spatřil v jejích očích sama sebe, uzavřeného ve dvou zářících kapkách průzračné tekutiny, sama sebe, černého a drobného, jemně do všech podrobností, s rýhou kolem úst. se vším všudy, jako by její oči byly dva zázračné kousky jantaru a mohly ho oblít a nedotčeného uchovat. Její obličej, obrácený teď k němu, byl z křehkého mléčného skla a prosvítalo jím měkké nezhasínající světlo. Nebylo to hysterické světlo elektrické žárovky, ale - co? Podivně příjemné, vzácné a lichotivé světlo svíčky. Kdysi dávno, ještě když byl malý, byla jednou v elektrárně porucha. Matka tehdy našla a rozsvítila poslední svíčku a na hodinku znovu objevili osvětlení, v němž prostor ztratil svoje nesmírné rozměry, schoulil se útulně kolem nich a oni, matka a syn zůstali sami, změnění a doufající, že elektrárna ještě chvíli nebude fungovat...
    \newpage
    \section*{Analýza uměleckého textu}
    \begin{itemize}
        \item\textbf{zasazení výňatku do kontextu díla}
        \begin{itemize}
            \item úryvek je z první třetiny knihy
            \item popisuje jedno ze setkání Montaga a Clarissy, před tím než Clarissa zmizí a Montagovy začnou docházet problémy jeho světa
        \end{itemize}
        \item\textbf{téma a motiv}
        \begin{itemize}
            \item\textbf{téma: }úpadek společnosti
            \item\textbf{motivy: }pálení knih požárníky, ztráta schopnosti myslet, manipulace, strach, rychlost, všichni mají být stejní
        \end{itemize}
        \item\textbf{časoprostor}
        \begin{itemize}
            \item\textbf{čas: }blíže neurčená budoucnost (2. polovina 20. století, nebo později) - antiutopická vize
            \item\textbf{prostor: }USA, blíže neurčeno (víme, že proběhla válka Severu proti Jihu)
        \end{itemize}
        \item\textbf{kompoziční výstavba}
        \begin{itemize}
            \item Členěno do tří podobně dlouhých částí, podle Montagova postoji vůči povolání a společnosti
            \item Části nejsou dále děleny do kapitol
            \item celková kompozice je chronologická
            \item jsou zde retrospektivní úseky - Když Beatty vysvětluje Montagovi historii hasičského povolání, nebo když Faber popisuje historii z pohledu bývalého učitele.
        \end{itemize}
        \item\textbf{literární žánr a druh}
        \begin{itemize}
            \item\textbf{druh:} epika, próza
            \item\textbf{žánr:} antiutopický román
        \end{itemize}
        \item\textbf{vypravěč / lyrický subjekt}
        \begin{itemize}
            \item nadosobní vševědoucí vypravěč
            \item er-forma
        \end{itemize}
        \item\textbf{postava}
        \begin{itemize}
            \item\textbf{Guy Montag (hlavní postava) -} požárník, nemá za úkol hasit požáry, ale naopak je zakládat, když se u někoho najdou nelegálně držené knihy. Vývojová postava - Zpočátku dělá své povolání s nadšením, ale postupem času mu začíná docházet, že je to špatně. Přemýšlivý, impulzivní, neposlušný.
            \item\textbf{Mildred Montagová (vedlejší postava) -} Guyova manželka, typický zástupce společnosti. Vede život zaměřený na rychlé prožitky, ne na intelektuální uspokojení, neustále sleduje nejrůznější televizní pořady na svých telestěnách. Je ustrašená, slabomyslná, záleží jí víc na virtuálních kamarádkách, než na manželovi. Nakonec na Guye zavolá požárníky.
            \item\textbf{kapitán Beatty (hlavní postava) -} hlavní požárník, na rozdíl od řadových požárníků má přehled o literatuře a tím co bylo před počátkem požárníků. Přesvědčivý, manipulátor, všímavý. Už od počátku si všimne Montagova vnitřního boje.
            \item\textbf{Faber (hlavní postava) -} bývalý profesor filozofie, skrývá se. Zbabělý cokoli udělat proti systému, dokud se nesetká s Montagem. Poté pomůže Montagovi s útěkem.
        \end{itemize}
        \item\textbf{typy promluv}
        \begin{itemize}
            \item přímá vlastní řeč - \textit{"Už vím," řekl, "vy jste naše nová sousedka."}
            \item neznačená přímá řeč - \textit{Nebylo to hysterické světlo elektrické žárovky, ale - co?}
        \end{itemize}
        \item\textbf{jazykové prostředky a jejich funkce ve výňatku}
        \begin{itemize}
            \item spisovný jazyk
            \item v přímé řeči jednoduché věty, nebo krátká souvětí - promluvy postav zní autentičtěji
            \item ve vypravěčových pasážích poměrně dlouhá souvětí
        \end{itemize}
        \item\textbf{tropy a figury a jejich funkce ve výňatku}
        \begin{itemize}
            \item text neobsahuje moc figur ani tropů
            \item\textbf{figury}
            \begin{itemize}
                \item epizeuxis: Poznala bych - poznala bych
            \end{itemize}
            \item\textbf{tropy}
            \begin{itemize}
                \item přirovnání: její tvář svítila jako sníh
                \item personifikace: šaty šeptaly
                \item oxymoron: suchý déšť
            \end{itemize}
        \end{itemize}
    \end{itemize}
    \section*{Literárně historický kontext a autor}
    R.D.Bradbury byl americký spisovatel, tvořil především v 2. polovině 20. století.
    Známý autor antiutopií, sci-fi a filozofických děl.
    Inspirací mu byla díla Verna a Welse, která četl v dětství.
    Kladl důraz na duchovno a varoval před příliš velkým rozvojem techniky.
    Předpověděl mnoho věcí do budoucna - "zrychlování" světa, úpadek zájmu o knihy a mentální rozvoj.
    Je to jedno z prvních děl, napsal ho přibližně ve 30 letech, přesto zaznamenalo velký úspěch a je jedním z jeho nejznámějších.
    Další dílo s podobnou tématikou je Marťanská kronika, soubor povídek o lidské kolonizaci Marsu a vyhlazení původních Marťanů.

    Autoři s podobnou tvorbou jsou například George Orwell nebo Arthur C. Clarke.
    Dílo bylo přijato dobře, bylo 2x zfilmováno.
    \section*{Zdroje}
    \begin{verbatim}
        Čtenářský deník nejen k maturitě, Jiří Mrákota - vydavatelství jazykové literatury
        Literatura v kostce pro SŠ, Marie Sochrová
        https://en.wikipedia.org/wiki/Fahrenheit_451
        https://en.wikipedia.org/wiki/Ray_Bradbury
    \end{verbatim}
    \end{document}