\documentclass[11pt]{article}
\usepackage[a4paper, total={18cm, 27cm}]{geometry}
\usepackage[czech]{babel}
\usepackage[utf8]{inputenc}

\begin{document}
    \begin{center}
        \underline{\textbf{\Huge Václav Havel - Zahradní slavnost}}
    \end{center}
    \section*{Úryvek}
    HUGO Pseudofamiliární zahajovačské frazeologie, ukrývající za rutinou profesionálního humanismu hlubokou názorovou rozbředlost, což dovedlo nakonec Zahajovačskou službu zákonitě do situace podkopávače pozitivního konsolidačního úsilí Likvidačního úřadu, jehož historicky naprosto nutným výrazem je moudrý akt její likvidace!

    ŘEDITEL Absolutně souhlasím!

HUGO Vy pořád jen souhlasíte – a skutek utek! Ne, ne, takhle tu likvidaci nikdy neuděláme! Čas letí – přineste mi kafe!

ŘEDITEL Promiňte, ale –

HUGO Opravdu nevím, o jakém „ale“ tu ještě mluvíte –

ŘEDITEL Nemluvím o „ale“, ale chci říct –

HUGO Tak vy chcete „ale“ dokonce říct?

ŘEDITEL Nechci říct „ale“, ale –

HUGO Možná nechcete říct „ale, ale“, ale chcete říct „ale“, a to úplně stačí! Mě ale žádným „ale“ neoalíte!

ŘEDITEL Promiňte, ale – kolik chcete kostek?

HUGO Čtyřiadvacet. A už to dál nerozpatlávejte, teď není čas na slovní hříčky!

TAJEMNÍK Dobrý den! Tak se dáme do toho, ne?

HUGO Samozřejmě. Kde je tady trezor?

TAJEMNÍK To musíte vědět vy!

HUGO Vy tady neděláte? Dobrý den!

TAJEMNÍK Chci tady dělat –

HUGO A kde děláte?

TAJEMNÍK Na Likvidačním úřadě – (Podává mu ruku) Josef Doležal.

HUGO A chcete dělat tady?

TAJEMNÍK Musím tady dělat –

HUGO On se Likvidační úřad likviduje? Hugo Pludek.

TAJEMNÍK Copak on se má Likvidační úřad likvidovat? Josef Doležal.

HUGO (podává mu ruku) Hugo Pludek. Nebo snad myslíte, že by se neměl likvidovat?

TAJEMNÍK No promiňte! Všichni přece dobře víme, že Likvidační úřad je přežitkem minulosti! A i když nelze popřít, že v údobí boje proti některým neuváženým výstřelkům v činnosti Zahajovačské služby sehrál Likvidační úřad některými moudrými likvidačními zákroky nesporně pozitivní úlohu, přesto by upadal do sentimentálního staromilství –

HUGO A byrokratického konzervatismu ten, kdo by –

TAJEMNÍK Neviděl práci Likvidačního úřadu v perspektivě jeho pozdějšího vývoje, kdy četnými neuváženými likvidačními zákroky proti mnoha pozitivním prvkům v práci Zahajovačské služby –

HUGO Sehrál Likvidační úřad nesporně negativní úlohu, k čemuž došlo v souvislosti s činností některých likvidačních úředníků –

TAJEMNÍK Kteří postupně nadřadili –

TAJEMNÍK a HUGO (společně) Administrativní stránku likvidační praxe jejímu společenskému obsahu, čímž dostala činnost Likvidačního úřadu nezdravě samoúčelný charakter, neboť tím byla násilně odtržena od života –
    \section*{Analýza uměleckého textu}
    \begin{itemize}
        \item\textbf{zasazení výňatku do kontextu díla}
        \begin{itemize}
            \item 3. dějství, Hugo je na likvidačním úřadě a připravuje jeho likvidaci
        \end{itemize}
        \item\textbf{téma a motiv}
        \begin{itemize}
            \item\textbf{téma: }absurdita systému (přizpůsobování se vnější moci - konformismus, morální úpadek), byrokratizace, absurdita jazyka, ztráta identity
            \item\textbf{motivy: }opakující se vyprázdněné fráze, Hugo hraje šachy sám se sebou, zahajovací a likvidační úřad
        \end{itemize}
        \item\textbf{časoprostor}
        \begin{itemize}
            \item\textbf{čas: }1 den, blíže neurčený - nehraje to roli (nejspíš v době nějakého totalitního režimu)
            \item\textbf{prostor: }reálný svět, blíže neurčený, také nehraje roli
        \end{itemize}
        \item\textbf{kompoziční výstavba}
        \begin{itemize}
            \item věnováno Janu Grossmanovi
            \item hra je dělena do 4 dějství (nerespektuje klasické rozdělení)
            \item chronologická
        \end{itemize}
        \item\textbf{literární žánr a druh}
        \begin{itemize}
            \item\textbf{druh: }drama
            \item\textbf{žánr: }absurdní drama, komedie
        \end{itemize}
        \item\textbf{vypravěč / lyrický subjekt}
        \begin{itemize}
            \item vypravěč není, je to drama, pouze scénické poznámky
            \item du-forma
        \end{itemize}
        \item\textbf{postava}
        \begin{itemize}
            \item Hugo Pludek - hlavní, přirozená, fiktivní, vývojová, neutrální - inteligentní, chápavý mladík, naděje své průměrné rodiny - jeho bratr je buržoazní intelektuál (nemá budoucnost v režimu) - hraje šachy sám se sebou, rodiče ho posílají na zahradní slavnost likvidačního úřadu, mistrně ovládne vyprázdněné fráze a pomocí jejich dobrého užívání stoupá v žebříčku funkcionářů (nejsou schopni mu oponovat v argumentech), ale ztrácí tím vlastní identitu - když se vrátí domů, rodiče ho ani nepoznají
            \item vystupují různí funkcionáři likvidačního a zahajovacího úřadu
        \end{itemize}
        \item\textbf{vyprávěcí způsoby}
        \begin{itemize}
            \item du-forma, repliky, dialogy
        \end{itemize}
        \item\textbf{typy promluv}
        \begin{itemize}
            \item dialogy (hra je založena na nedorozumění)
        \end{itemize}
        \item\textbf{jazykové prostředky a jejich funkce ve výňatku}
        \begin{itemize}
            \item spisovný
            \item Plzák nespisovný - nářečí, gramatické chyby
            \item jazyk je prostředkem nedorozumění
            \item neustálé opakování frází, komolení úsloví, využití obecné češtiny (projev nevzdělanosti ?)
        \end{itemize}
        \item\textbf{tropy a figury a jejich funkce ve výňatku}
        \begin{itemize}
            \item personifikace - čas letí
            \item celé je to metafora na nedohranou šachovou partii
        \end{itemize}
    \end{itemize}
    \section*{Literárně historický kontext a autor}
    Václav Havel, poslední prezident Československa a první prezident České republiky, byl světově proslulý dramatik a významný esejista.
    Známý po celém světě, několikrát nominovaný na Nobelovu cenu míru.
    Dostudoval až v 60. letech - byl buržoazního původu.
    Živil se jako kulisák, pak přišel nečekaný obrovský úspěch Zahradní slavnosti.
    V 60. letech tvořil podobné hry jako Zahradní slavnost, např. Vyrozumění - kritika jazyka, vymyšlený jazyk Ptydepe. Věnoval se i experimentální poezii - Antikódy (typogramy).
    Založil samizdatové nakladatelství Expedice a Chartu 77 (reakce na zatčení skupiny Plastic people of the Universe, porušování lidských práv).
    Změnil svou poetiku - více autobiografické vaňkovské hry - zobrazují člověka intelektuála perzekuovaného režimem - např. Audience.
    Významný esejista - Dopis Gustavu Husákovi (otevřený dopis, perfektní analýza strachu a přetvářky lidí v režimu), Moc bezmocných.
    Tvořil v 2. polovině 20.století a na začátku 21. století.
    Autoři ze stejného období - Ludvík Vaculík, Hrabal, Hrubín, Klíma, ve světě - Beckett, Ionesco (absurdní drama).
    Díla zaznamenala obrovský úspěch po celém světě. Hra nebyla zfilmována.
\end{document}